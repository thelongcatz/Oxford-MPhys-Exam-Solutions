\draft
\begin{parts}
	\part
	\begin{subparts}
		\subpart Partial width refers to the contribution to the total width due to a particular process $i$, e.g. $\Gamma_i$ is the width due to the initial state and $\Gamma_f$ is that due to the final state in Breit-Wigner equation.
		
		Total width is simply the sum of all partial widths, and is the FWHM one measures from data: $\Gamma = \sum_j \Gamma_j$
		{\color{red}Circular definition, what do the partial/total widths correspond to physically?}
		
		For Breit-Wigner resonance to occur, the incoming wave must be plane wave, and that the time spent in the interaction potential is long enough that the probability of transitions is random.
		
		\subpart At resonance,
		\begin{align}
			\sigma_\textnormal{peak} &= \pi g \rbracket{\frac{\lambda}{2\pi}}^2 \frac{\Gamma_i \Gamma_f}{\frac{1}{4}\Gamma^2} \notag \\
			&= \frac{g\lambda^2}{\pi} \frac{\Gamma_i \Gamma_f}{\Gamma^2}
			\label{eqn:q4-peak-1}
		\end{align}
		
		For FWHM $\Gamma = 2\delta$,
		\begin{align}
			\frac{1}{2} \sigma_\textnormal{peak} &= \pi g \rbracket{\frac{\lambda}{2\pi}}^2 \frac{\Gamma_i \Gamma_f}{\delta^2 + \frac{1}{4}\Gamma^2} \notag \\
			\Rightarrow \sigma_\textnormal{peak} &= \frac{g\lambda^2}{2\pi} \frac{\Gamma_i \Gamma_f}{\delta^2 + \frac{1}{4}\Gamma^2}
			\label{eqn:q4-peak-2}
		\end{align}
		
		From \eqref{eqn:q4-peak-1}, neglecting other channels so $\Gamma = \Gamma_i + \Gamma_f$:
		\begin{equation*}
			\sigma_\textnormal{peak} = \frac{g\lambda^2}{\pi} \frac{\Gamma_i (\Gamma - \Gamma_i)}{\Gamma^2}
		\end{equation*}
		
		\begin{align*}
			g &= \frac{2 \cdot 1 + 1}{(2 \cdot \frac{3}{2} + 1)(2 \cdot \frac{1}{2} + 1)} \\
			&= \frac{1}{8}
		\end{align*}
		
		\begin{align*}
			p &= \frac{h}{\lambda} \\
			\Rightarrow \lambda &= \frac{h}{p} \mtext{where $p$ is proton momentum} \\
			T &= \frac{p^2}{2\massproton} \\
			\Rightarrow p &= \sqrt{2 \massproton T} \mtext{at non-relativistic regime} \\
			\Rightarrow \lambda &= \frac{h}{\sqrt{2\massproton T}} \\
			&= \frac{2\pi\hbar c}{\sqrt{2\massproton c^2 T}} \\
			&= \frac{2\pi (\SI{197.33}{\mega\electronvolt\fermi})}{\sqrt{2(\SI{938.3}{\mega\electronvolt})(\SI{1.4}{\mega\electronvolt})}} \\
			&= \SI{24.2}{\fermi}
		\end{align*}
		
		So:
		\begin{align*}
			\sigma_\textnormal{peak} \cdot \pi \cdot \Gamma^2 &= g\lambda^2 \Gamma_i (\Gamma - \Gamma_i) \\
			\Rightarrow g\lambda^2 \Gamma_i^2 - g\lambda^2 \Gamma \Gamma_i + \Gamma^2 \pi \sigma_\textnormal{peak} &= 0 \\
			\Gamma_i &= \frac{g\lambda^2 \Gamma \pm \sqrt{g^2 \lambda^4 \Gamma^2 - 4g\lambda^2 \Gamma^2 \pi \sigma_\textnormal{peak}}}{2g\lambda^2} \\
			&= \num{1.059}\Gamma \mtext{or\hspace{1em}} \num{-0.059}\Gamma \mtext{(unphysical)} \\
			&= \SI{1.218}{\mega\electronvolt} \mtext{or\hspace{1em}} \SI{-0.06785}{\mega\electronvolt}
		\end{align*}
		{\color{red}Doesn't make sense as $\Gamma_i + \Gamma_f = \Gamma > \Gamma_i$}
		
		\subpart Instead of $\alpha$ decay, one might also observe $\beta$ decay and $\gamma$ emission.
	\end{subparts}
	
	\part Gamow factor $G$ encodes the probability of an alpha particle tunneling through the Coulombic barrier of the nucleus, $e^{2G}$.
	
	Now consider the Coulomb potential of the nucleus sans $\alpha$-particle: $\dfrac{(Z-2)e}{4\pi\permittivity r}$
	
	So the energy barrier required would be:
	\begin{equation*}
		V \sim \frac{2(Z-2)e^2}{4\pi\permittivity r}
	\end{equation*}
	with $r \sim r_0 A^{1/3}$ where $r_0 \simeq \SI{1.2}{\fermi}$ and $A$ is the atomic \#.
	
	So:
	\begin{equation*}
		V = \frac{2(Z-2)e^2}{4\pi\permittivity r_0 A^{1/3}}
	\end{equation*}
	
	For \element{Be}{8}, $Z=4$, $A=8$:
	\begin{align*}
		\Rightarrow V &= \frac{4e^2}{4\pi\permittivity r_0 \underbracket{8^{1/3}}_{2}} \\
		&= \frac{2e^2}{4\pi\permittivity r_0} \\
		&= \frac{2\alpha\hbar c}{r_0} \\
		&= \frac{2(\frac{1}{137})(\SI{197.33}{\mega\electronvolt\fermi})}{\SI{1.2}{\fermi}} \\
		&= \SI{2.4}{\mega\electronvolt} \gg Q = \SI{91}{\kilo\electronvolt}
	\end{align*}
	
	\begin{align*}
		G &= \sqrt{\frac{2m_\alpha}{\hbar^2}} \indefint{\sqrt{V(r) - Q}}{r} \\
		&\simeq \sqrt{\frac{2m_\alpha}{\hbar^2}} \defint{0}{\infty}{\sqrt{\frac{4e^2}{4\pi\permittivity r}}}{r} \\
		&= \sqrt{\frac{2m_\alpha}{\hbar^2}} \defint{0}{\infty}{\sqrt{\frac{4\alpha\hbar c}{r}}}{r} \\
		&= \sqrt{\frac{2m_\alpha}{\hbar^2}} \sqrt{4\alpha\hbar c} \sbracket{r^{1/2}}_{r=0}^\infty
	\end{align*}
\end{parts}