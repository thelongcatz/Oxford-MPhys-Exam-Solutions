\draft
\begin{parts}
	\part
	\begin{subparts}
		\subpart The formation of $\positron \electron \to \muonp \muonn$ is fixed within the probed energy range (EM channel only).
		So the $R$ ratio can offer insights into quark formation.
		
		As the probed range is lower than masses of the W and Z gauge bosons, the quarks can only form via EM interaction, hadronisation occurs after the pair formation with gluon emission.
		{\color{red}W and Z can be virtual, and gluons are strong interaction.}
		
		Note that the graph has 2 plateaux, and this is due to the additional formation of $\Qcharm$ and $\Qbottom$ quarks respectively.
		Hence the existence of quark is demonstratable.
		{\color{red}Can you prove the CoM energy of the plateau is correct for these?}
		
		The idea of colour may also be validated by measuring the increase in pluteau height and comparing that against the expected increase, there should he a degeneray of 3 encoded in the hadronisation.
		
		{\color{red}8 marks so more details.}
		
		\subpart Width of the peak $\Gamma$ (FWHM): $\Gamma \simeq \SI{0.4}{\giga\electronvolt}$
		
		Lifetime:
		\begin{align*}
			\tau &= \frac{\hbar}{\Gamma} \\
			&= \frac{\hbar c}{\Gamma c} \\
			&= \frac{\SI{197.33}{\mega\electronvolt\fermi}}{\SI{400}{\mega\electronvolt} \cdot \SI{3e8}{\metre\per\second}} \\
			&= \SI{1.64e-24}{\second}
		\end{align*}
		{\color{red}Seems very low?}
	\end{subparts}
	
	\part As the available energy increases, more formation would be possible as the masses of the gauge bosons are overcome:
	\begin{itemize}
		\item At around \SI{90}{\giga\electronvolt}, $\Zz$ formation becomes possible and $R$ should double as $\Zz$ couples all fermions (in pairs).
		\item At around \SI{125}{\giga\electronvolt}, $\higgs$ formation becomes possible and $R$ should further double for the same reason.
	\end{itemize}
	{\color{red}Some calculations to demo $R$ of formed particles.}
	
	Apart from having new formation channels, we should also observe a jump in the plateau in several spots where new quarks are formed:
	\begin{itemize}
		\item Around \SI{173}{\giga\electronvolt} due to the $\Qtop$ quark.
	\end{itemize}
	
	\image{.8\linewidth}{q3-r-ratio}
	
	\part Practically at high energies, $\positron$ and $\electron$ would emit bremsstrahlung due to the acceleration in synchrotron (required since collision between particles is difficult so need multiple rounds) and shed away its energy, which in turn requires higher power to the boosters in the storage ring.
	
	{\color{red}Does cross section for collision.}
\end{parts}