\draft
\begin{parts}
	\part
	\begin{subparts}
		\subpart A nuclear form factor is an additional factor to the scattering amplitude.
		It enters through the matrix element to encode the spatial extent of a charge distribution (i.e. the Fourier transform of the charge distribution).
		
		\subpart Spherically symmetric charge distribution has density:
		\begin{equation*}
			\rho = \frac{q}{\frac{4}{3}\pi r^3} = \frac{3q}{4\pi r^3}
		\end{equation*}
		
		\begin{align*}
			F(k) &= \defint{-\infty}{\infty}{\rho e^{-i\mathbf{k}\cdot\mathbf{x}}}{{}^3 \mathbf{x}} \\
			&= \defint{0}{2\pi}{}{\phi} \defint{0}{\pi}{}{\theta} \defint{0}{\infty}{\frac{3q}{4\pi r} r^2 \sin\theta e^{ikr\cos\theta}}{r} \\
			&= 2\pi \defint{0}{\infty}{}{r} \defint{0}{\pi}{\frac{3q}{4\pi r} \sin\theta e^{ikr\cos\theta}}{\theta}
		\end{align*}
		
		Substituting $u=\cos\theta \Rightarrow \inftsml{u}=-\sin\theta\inftsml{\theta}$, the limits map from $[0, \pi] \to [1, -1]$:
		\begin{align*}
			\Rightarrow F(k) &= 2\pi \defint{0}{\infty}{\frac{3q}{4\pi r}}{r} \defint{-1}{1}{e^{ikru}}{u} \\
			&= 2\pi \defint{0}{\infty}{\frac{3q}{4\pi r}}{r} \sbracket{\frac{e^{ikru}}{ikr}}_{u=-1}^1 \\
			&= 2\pi \defint{0}{\infty}{\frac{3q}{4\pi r} \cdot \frac{1}{kr} \cdot \underbracket{\frac{e^{ikr} - e^{-ikr}}{i}}_{2\sin (kr)}}{r} \\
			&= \frac{4\pi}{k} \defint{0}{\infty}{\frac{3q}{4\pi r^2} \sin (kr)}{r} \\
			&= -\frac{4\pi}{k} \defint{0}{\infty}{r \rho (r) \sin (kr)}{r}
		\end{align*}
		where $k$ is the scattering wavevector, $\rho$ is the charge density. (sign check)
		
		For uniformly charge sphere of radius $R$,
		\begin{align*}
			F(k) &= \frac{4\pi}{k} \defint{0}{R}{r \frac{3q}{4\pi R^3} \sin (kr)}{r} \\
			&= \frac{3q}{kR^3} \defint{0}{R}{r \sin (kr)}{r}
		\end{align*}
		
		Integration by parts:
		\begin{equation*}
			\begin{matrix}
				A &= r \\
				\inftsml{B} &= \sin (kr) \inftsml{r}
			\end{matrix}
			\Rightarrow
			\begin{matrix}
				\inftsml{A} &= \inftsml{r} \\
				B &= -\frac{1}{k} \cos (kr)
			\end{matrix}
		\end{equation*}
		
		\begin{align*}
			\Rightarrow F(k) &= \frac{3q}{kR^3} \cbracket{\sbracket{-\frac{r}{k} \cos (kr)}_{r=0}^R + \frac{1}{k} \defint{0}{R}{\cos (kr)}{r}} \\
			&= \frac{3q}{kR^3} \cbracket{-\frac{R}{k} \cos (kR) + \frac{1}{k^2} \sin (kR)} \\
			&= 3q \cbracket{-\frac{1}{(kR)^2} \cos (kR) + \frac{1}{(kR)^3} \sin (kR)}
		\end{align*}
		So $C=3q$, $x=kR$. (Should be $e^{-ikx}$, perhaps wrong FT)
	\end{subparts}
	
	\part
	\begin{subparts}
		\subpart For $F(k)$ to be $0$ (scattering amplitude without $F(k)$ drops slowly),
		\begin{align*}
			\frac{1}{(kR)^2} \cos (kR) &= \frac{1}{(kR)^3} \sin (kR) \\
			\Rightarrow \tan (kR) &= kR
		\end{align*}
		
		Now $\mathbf{k} = \mathbf{G}^\prime - \mathbf{G} \Rightarrow k = 2G \sin \frac{\theta}{2}$ with total scattering angle $\theta$.
		\image{.15\linewidth}{q1-scattering-angle}
		
		Incoming wavevector $\mathbf{G} = \dfrac{\mathbf{p}}{\hbar}$ where $\mathbf{p}$ is electron momentum.
		Electron energy:
		\begin{align*}
			E &= \sqrt{(\masselectron c^2)^2 + (pc)^2} \\
			\Rightarrow pc &= \sqrt{E^2 - \masselectron^2 c^4} \\
			\Rightarrow p &= \frac{1}{c} \sqrt{(\SI{250}{\mega\electronvolt})^2 - (\SI{0.511}{\mega\electronvolt})^2} \\
			&= \SI{250}{\mega\electronvolt /c}
		\end{align*}
		Energy $\gg \masselectron c^2$, can argue this is true in ultra-relativistic limit.
		
		So $G=\dfrac{\SI{250}{\mega\electronvolt}}{\hbar c}=\SI{1.267}{\per\fermi}$.
		
		For $\theta = 53\degree$,
		\begin{align*}
			k &= 2G \sin \frac{53\degree}{2} \\
			&= 2(\SI{1.267}{\per\fermi}) \sin 26.5\degree \\
			&= \SI{1.131}{\per\fermi} \\
			\Rightarrow R &\simeq \SI{3.975}{\fermi} \mtext{by trial and error\hspace{1em}$\leftarrow$ Ca-48}
		\end{align*}
		
		For $\theta = 57\degree$,
		\begin{align*}
			k &= \SI{1.209}{\per\fermi} \\
			\Rightarrow R &= \SI{3.720}{\fermi} \mtext{$\leftarrow$ Ca-40}
		\end{align*}
		
		\subpart The liquid drop model states that nuclear radius $r = r_0 A^{1/3}$ where $r_0 = \SI{1.2}{\fermi}$, $A$ is atomic number.
		
		For Ca-40, $r \simeq \SI{4.10}{\fermi}$ $\rightarrow$ \% error: $9.3\%$
		
		For Ca-48, $r \simeq \SI{4.36}{\fermi}$ $\rightarrow$ \% error: $8.8\%$
		
		So there exists significant discrepancy between the experimental results and the liquid drop model, this is likely due to the effects of the shell model.
	\end{subparts}
\end{parts}