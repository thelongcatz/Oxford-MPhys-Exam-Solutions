\draft
\begin{parts}
	\part Consider the SEMF:
	\begin{equation*}
		B(Z, A) = \underbracket{a_V A - a_S A^{2/3}}_{\textnormal{liquid-drop model}} - \underbracket{a_C \frac{Z^2}{A^{1/3}}}_{\textnormal{Coulomb repulsion}} - \underbracket{a_A \frac{(A-2Z)^2}{A}}_{\textnormal{asymmtery}} + \delta (A, Z)
	\end{equation*}
	where
	\begin{equation*}
		\delta (A, Z) =
		\begin{cases}
			\frac{\SI{-12}{\mega\electronvolt}}{\sqrt{A}} \mtext{o-o} \\
			0 \mtext{o-e/e-o} \\
			\frac{\SI{+12}{\mega\electronvolt}}{\sqrt{A}} \mtext{e-e}
		\end{cases}
	\end{equation*}
	
	We then seek the stationary point:
	\begin{align*}
		\rbracket{\pderi{B}{Z}}_A &= 0 \\
		\Rightarrow Z &= \frac{2a_A A}{a_C A^{2/3} + 4a_A}
	\end{align*}
	where $a_C \sim \SI{1}{\mega\electronvolt}$, $a_A \sim \SI{10}{\mega\electronvolt}$.
	
	Recall that the nucleons are fermions, which abide Pauli's exclusion principle.
	Therefore it is more energetically favourable to have approximately equal number of protons and neutrons than letting one type pile up on the energy level.
	\begin{equation*}
		\lim_{A \to 0} Z = \frac{A}{2}
	\end{equation*}
	
	However, this scenario begins to break down for high $Z$ nuclei as electromagnetic repulsion between protons begins to dominate over strong interaction, meaning that the balance will start to tip towards having more neutrons.
	\begin{equation*}
		\lim_{A \to \infty} = \frac{2a_A}{a_C} A^{1/3}
	\end{equation*}
	
	In $\beta^-$ decay, also for a fission free neutron, a neutron decays into a proton, an electron and an anti-electron neutrino.
	\begin{equation*}
		\neutron \to \proton + \electron + \aneutrinoe
	\end{equation*}
	
	The reason why an antineutrino is produced has to do with the conservation of lepton numbers, which is a consequence of $\mathrm{W}^\pm$ coupling.
	\image{.2\linewidth}{q1-w-e}
	And so $0 = (+1) + (-1)$ since $\electron$ has lepton number $+1$, $\aneutrinoe$ has $-1$.
	
	As $A$ decreases, we would have more $\neutron$ than $\proton$ but need $Z \simeq \dfrac{A}{2}$ so $\beta^-$ decay occurs.
	
	\part Inverse beta decay:
	\begin{equation*}
		\aneutrinoe + \proton \to \positron + \neutron
	\end{equation*}
	\image{.5\linewidth}{q1-inverse-beta-decay}
	
	Recall that from Fermi's Golden Rule that
	\begin{equation*}
		\Gamma = \frac{2\pi}{\hbar} \abs{M_{if}}^2 \rho (E)
	\end{equation*}
	and matrix element $M_{if}$ is dependent on the momentum transfer by the gauge boson. (how?)
	
	Hence it is expected that the reaction cross-section, which is proportional to $\Gamma$ the reaction rate, to increase with greater $\aneutrinoe$ energy. (details needed)
	
	Sargent's rule: $\Gamma \propto Q^5$
	\image{.2\linewidth}{q1-sargent}
	D.O.S.:
	\begin{align*}
		\inftsml{{}^2 N} &= \frac{V \inftsml{\Omega} P^2 \inftsml{P}}{h^3} \frac{V \inftsml{\Omega} Q^2 \inftsml{Q}}{h^3} \\
		&= \frac{1}{4\pi^4} \frac{1}{h^6} p^2 q^2 \inftsml{P} \inftsml{Q} \\
		q &= \frac{(Q - E_\nu)}{c} \mtext{for } Q \gg \masselectron c^2
	\end{align*}
	
	\part Detected interactions in total:
	\begin{equation*}
		\deri{I}{\mathrm{E}_\nu} = \sigma_\nu \rho (E)
	\end{equation*}
	since reaction rate $\Gamma \propto \rho (E)$ and $\sigma_\nu$.
	
	Detected reaction rate $\propto \sigma_\nu$ since cross section dictates the total reaction rate (why?)
	
	$\propto \diagderi{N_\nu}{E_\nu} \inftsml{E_\nu}$ as a result of thermodynamical distribution (select population with specific energy for specific $\sigma_\nu$).
	
	{
		\color{red} $\sigma_{\bar{\nu}} \deri{N_{\bar{\nu}}}{E_{\bar{\nu}}}$ represents the cross section as a function of energy combined with the distribution of $\aneutrinoe$ energies per fission reaction, i.e. total cross section of all $\aneutrinoe$ produced at each energy per fission reaction.
		We have to integrate to get total cross section for all energies and particles.
		\begin{equation*}
			\underbracket{\sigma \inftsml{E}}_{\substack{\textnormal{differential}\\\textnormal{cross section}\\\textnormal{for each $E$}}} \cdot \underbracket{\deri{N}{E}}_{\textnormal{\# of $\nu$ for each $E$}}
		\end{equation*}
		\image{.6\linewidth}{q1-xsection}
		\begin{align*}
			\sigma_{\bar{\nu}} &= \frac{\textnormal{No. events $t$}-1}{\textnormal{Neutrino flux}} \\
			\therefore \sigma_{\bar{\nu}} \inftsml{N_{\bar{\nu}}} &= \frac{\textnormal{No. of events $t$}-1}{\textnormal{Unit area}} \propto \textnormal{Rate of observed $\bar{\nu}$ interactions}
		\end{align*}
	}
	
	\part \begin{subparts}
		\subpart Generator produces \SI{100}{\mega\watt} and each \element{U}{235} fission releases \SI{210}{\mega\electronvolt}:
		\begin{equation*}
			\therefore \textnormal{rate} = \frac{\textnormal{power}}{\textnormal{Energy per fission}} = \SI{2.97e18}{\per\second}
		\end{equation*}
		
		\subpart Approximate $\sigma \diagderi{N}{E}$ as a Gaussian, we can derive mean and s.d. from the data:
		\begin{equation*}
			I = \SI{0.669e-42}{\centi\metre\squared\per fission}
		\end{equation*}
		
		\subpart \image{.2\linewidth}{q1-photodetector-setup}
		We have the rate given by:
		\begin{equation*}
			\textnormal{Rate} = \rho \sigma v
		\end{equation*}
		where $\rho$ is the density of detector medium, $\sigma$ is calculated in II per fission, $v=c$ for massless neutrinos.
		\begin{align*}
			\textnormal{Rate} &= \SI{0.025}{\per\second} \\
			&= \frac{\overbracket{N}^{\mathclap{\substack{\textnormal{\# of}\\\textnormal{scintillating particles}}}}}{\underbracket{V}_{\mathclap{\frac{4}{3}\pi(7)^3}}} \times \SI{0.669e-42}{\centi\metre\squared\per fission} \times \SI{2.97e18}{fission \per\second} \times \underbracket{L}_{\mathclap{\SI{7}{\metre}}}
		\end{align*}
		So
		\begin{align*}
			\rho &= N \times M_{\textnormal{C\textsubscript{9}H\textsubscript{10}}} \\
			&= \SI{6085}{\kilo\gram}
		\end{align*}
	\end{subparts}
\end{parts}