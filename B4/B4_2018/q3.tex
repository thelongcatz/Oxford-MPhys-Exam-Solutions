\draft
\begin{parts}
	\part $\mathbb{B}$ is defined as $+\frac{1}{3}$ for quarks, $-\frac{1}{3}$ for anti-quarks, $0$ for leptons.
	Therefore a baryon would have $\mathbb{B}=\pm 1$, where negative sign is for anti-baryons.
	Mesons only have $\mathbb{B}=0$ since they contain a quark-antiquark pair.
	{\color{red} More detailed explanation for baryons and mesons would be good.}
	
	\part \begin{subparts}
		\subpart
		\begin{align*}
			\proton + \proton &\to \neutron + \proton + \pip \\
			\begin{matrix}
				\mathbb{B} = 2 \\
				\mathbb{Q} = +2 \\
				\mathbb{L} = 0
			\end{matrix}
			&\phantom{\to}
			\begin{matrix}
				\mathbb{B} = 2 \\
				\mathbb{Q} = +2 \\
				\mathbb{L} = 0
			\end{matrix}
		\end{align*}
		As $\mathbb{B}$, $\mathbb{Q}$, $\mathbb{L}$ are all conserved, the above process is allowed. (Try breakdown the calc.)
		
		Feynman diagram:
		\image{.5\linewidth}{q3-pp}
		
		\subpart
		\begin{align*}
			\proton + \electron &\to \rhoz \\
			\begin{matrix}
				\mathbb{B} = 1 \\
				\mathbb{Q} = 0 \\
				\mathbb{L} = +1
			\end{matrix}
			&\phantom{\to}
			\begin{matrix}
				\mathbb{B} = 0 \\
				\mathbb{Q} = 0 \\
				\mathbb{L} = 0
			\end{matrix}
		\end{align*}
		Since lepton number and baryon number are not conserved, the process is forbidden.
		
		\subpart
		\begin{align*}
			\muonn &\to \electron + \neutrinom + \aneutrinoe \\
			\begin{matrix}
				\mathbb{B} = 0 \\
				\mathbb{Q} = -1 \\
				\mathbb{L} = 1
			\end{matrix}
			&\qquad
			\begin{matrix}
				\mathbb{B} = 0 \\
				\mathbb{Q} = -1 \\
				\mathbb{L} = 1
			\end{matrix}
		\end{align*}
		Since $\mathbb{B}$, $\mathbb{Q}$, $\mathbb{L}$ are all conserved, the process is allowed.
		
		Feynman diagram:
		\image{.5\linewidth}{q3-mu-e}
		
		\subpart
		\begin{align*}
			\neutrinoe + \positron &\to \Qdown + \bar{\Qup} \\
			\begin{matrix}
				\mathbb{B} = 0 \\
				\mathbb{Q} = +1 \\
				\mathbb{L} = 0
			\end{matrix}
			&\phantom{\to}
			\begin{matrix}
				\mathbb{B} = 0 \\
				\mathbb{Q} = -1 \\
				\mathbb{L} = 0
			\end{matrix}
		\end{align*}
		Since $\mathbb{Q}$ is not conserved, the process is forbidden.
	\end{subparts}
	
	\part X boson: $\mathbb{Q}=+\frac{1}{3}$; $\mathbb{B}$, $\mathbb{L}$ not conserved; $m_\mathrm{X} \gg \massproton$, $\mathbb{Q}$ conserved.
	
	Fundamental vertices:
	\image{.8\linewidth}{q3-fundamental-vertices}
	
	Now we try process $\proton \to \positron + \piz$:
	\image{.5\linewidth}{q3-px}
	
	$\neutron \to \positron + \pin$:
	\image{.5\linewidth}{q3-nx}
	
	\part Recall matrix element: (for $\proton \to \positron + \piz$)
	\begin{align*}
		|M_{if}| &= |\prod_i \textnormal{vertex factor}| \times |\prod_i \textnormal{propagator factor}| \\
		&= \lambda_q \lambda_l \cdot \abs{\frac{1}{\mathsf{P}^2 - m_\mathrm{X}^2 c^4}}
	\end{align*}
	with $\mathsf{P}^\mu = \rbracket{\dfrac{E_\Qup + E_\Qdown}{c}, \mathbf{p}_\Qup + \mathbf{p}_\Qdown}$ the 4-momentum transferred.
	
	But since $m_\mathrm{X} \gg \massproton$, $\mathsf{P}^2 \ll m_\mathrm{X}^2 c^2$ over most energy range in decay:
	\begin{equation*}
		\Rightarrow |M_{if}| \simeq \frac{\lambda_q \lambda_l}{m_\mathrm{X}^2 c^4}
	\end{equation*}
	
	Fermi's Golden Rule gives reaction rate:
	\begin{equation*}
		\Gamma = \frac{2\pi}{\hbar} |M_{if}|^2 \rho (E)
	\end{equation*}
	
	From Sargent's rule, density of states:
	\begin{equation*}
		\rho (E) \propto E^5 = (\massproton c^2)^5
	\end{equation*}
	
	Thus:
	\begin{equation*}
		\Gamma_{if} \propto \frac{2\pi}{\hbar} \frac{\lambda_l^2 \lambda_q^2}{m_\mathrm{X}^4} \massproton^5 c^2
	\end{equation*}
	
	So:
	\begin{align*}
		\tau_\mathrm{p} &\propto \frac{1}{\Gamma_{if}} \\
		&= A \frac{m_\mathrm{X}^4 \hbar}{\lambda_l^2 \lambda_q^2 \massproton^5 c^2}
	\end{align*}
\end{parts}