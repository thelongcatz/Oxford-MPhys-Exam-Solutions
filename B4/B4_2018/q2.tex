\draft
\begin{parts}
	\part In the light quark model, due to the invariance of strong force in the flavours of the quarks, we may propose a basis of representation as follows:
	\image{.5\linewidth}{q2-quark-isospin-strangeness}
	where $q$ is electric charge, $S$ is strangeness, $I_3$ is isospin.
	
	Assembling the hadrons with these quarks then yields the Eightfold Way, explaining the mass, charge, isospin and strangeness.
	
	For spin, recall that hadrons are composite particles of quarks, thus the quantum mechanical rule of spin addition holds, since each quark has spin $\frac{1}{2}$, a hadron may have spin $S=0, \frac{1}{2}, 1, \frac{3}{2}$.
	
	The parity of a particle is defined as $P=(-1)^L$ where $L$ is total orbital angular momentum.
	At ground state, $L=0 \Rightarrow P=\pm 1$.
	
	{
		\color{red} Give a more specific explanation for the multiplet given.
		
		Why, for example, does the $\SSigmaz$ have a strangeness of $-1$?
		What values of $q$ do the dashed lines represent?
	}
	
	\part \begin{equation*}
		\Kn + \proton \to \Omegan + \Kz + \mathrm{X}
	\end{equation*}
	
	Strong interaction preserves flavour so the quark contents should be preserved:
	\image{.45\linewidth}{q2-k-omega}
	So $\mathrm{X}$ is $\Kp$.
	
	\part For each strange quark added, the mass increases by \SI{147}{\mega\electronvolt /c^2}, so $\Omegan$ should have mass $\num{1532} + \num{147} = \SI{1679}{(\mega\electronvolt /c^2)}$.
	
	\image{.6\linewidth}{q2-momentum}
	At threshold energy, the products are at rest in ZMF.
	
	Lorentz invariance of system 4-momentum:
	\begin{align*}
		\mathsf{P}_\mu \mathsf{P}^\mu &= \mathsf{P}^\prime_\mu \mathsf{P}^{\prime^\mu} \\
		-\rbracket{\frac{E}{c} + \massproton c}^2 + p^2 &= -\rbracket{m_\Omega + m_\Kz + m_\mathrm{X}}^2 c^2 \\
		\underbracket{\rbracket{\frac{E^2}{c^2} - p^2}}_{m_\mathrm{K}^2 c^2} + 2E\massproton + \massproton^2 c^2 &= \rbracket{m_\Omega + m_\Kz + m_\mathrm{X}}^2 c^2 \\
		\Rightarrow E &= \frac{\sbracket{\rbracket{m_\Omega + m_\Kz + m_\mathrm{X}}^2 - \rbracket{m_\mathrm{K}^2 + \massproton^2}} c^2}{2\massproton} \\
		&= \SI{3200}{\mega\electronvolt}
	\end{align*}
	
	So threshold momentum:
	\begin{align*}
		pc &= \sqrt{E^2 - m_\mathrm{K}^2 c^4} \\
		\Rightarrow p &= \SI{3162}{\mega\electronvolt /c}
	\end{align*}
	
	\part \begin{align*}
		\Omegan &\to \Xiz + &\pin \\
		\Qstrange\Qstrange\Qstrange &\to \Qstrange\Qstrange\Qup &\Qdown\bar{\Qup}
	\end{align*}
	
	Conservation of flavour/quark number violated, try $\mathrm{W}^\pm$ interaction:
	\image{.6\linewidth}{q2-omega-decay}
	
	\part Momentum $p = \gamma mv = \SI{2015}{\mega\electronvolt /c}$. Hence energy:
	\begin{align*}
		E &= \sqrt{m^2 c^4 + p^2 c^2} = \gamma mc^2 \\
		\Rightarrow \gamma &= \frac{\sqrt{m^2 c^4 + p^2 c^2}}{mc^2} \\
		\Rightarrow v &= c\sbracket{1-\gamma^{-2}}^{1/2} \\
		&= c\sbracket{1-\frac{m^2 c^4}{m^2 c^4 + p^2 c^2}}^{1/2}
	\end{align*}
	
	Measured lifetime:
	\begin{align*}
		t &= \frac{d}{v} \\
		&= \frac{d}{c} \rbracket{\frac{m^2 c^4 + p^2 c^2}{m^2 c^4}}^{1/2}
	\end{align*}
	
	But time dilation means that $t = \gamma\tau$ where $\tau$ is proper lifetime:
	\begin{align*}
		\tau &= \frac{mc^2}{\sqrt{m^2 c^4 + p^2 c^2}} \cdot \frac{d}{c} \sbracket{\frac{m^2 c^4 + p^2 c^2}{p^2 c^2}}^{1/2} \\
		&= \frac{mcd}{pc} \\
		&= \frac{md}{p} \\
		&= \frac{\SI{0.0208}{\metre}}{c} = \SI{6.944e-11}{\second}
	\end{align*}
\end{parts}