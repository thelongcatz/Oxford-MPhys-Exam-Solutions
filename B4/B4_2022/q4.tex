\draft
\begin{parts}
	\part In the quark model, the baryon number is defined such that a baryon has $B=1$, an antibaryon has $B=-1$, i.e. a quark has $B=+\diagfrac{1}{3}$, an antiquark has $B=-\diagfrac{1}{3}$.
	This number arises from the fact that strong interaction conserves the quark number (gluons couple quark-antiquark so $\Delta B=0$).
	
	Strangeness is a label that mark the number of strange quarks a baryon contains, this is added to resolve the mystery of some long lived barons.
	$S-=1$\footnote{Or $S--$ if C++ is your cup of tea.} upon adding a strange quark.
	
	And $I_z$ is added in a similar fashion as angular momentum $J$ to mark the fact that strong interaction is invariant of quark flavour (originally marking neutron and proton as 2 states with $I=\diagfrac{1}{2}$, $I_z = \pm\diagfrac{1}{2}$).
	
	{\color{red}Draw 3-quark diagram of baryon octet.}
	
	\part Recall that baron and quarks are fermions, thereby obeying Pauli's Exclusion Principle.
	
	For $J=\diagfrac{1}{2}$ barons, the quarks should occupy the lowest energy shell, akin to the Rydberg model of hydrogen.
	However there only exists 2 possible ``slots'' for the same quark to lay in a shell since $I_z = \pm\diagfrac{1}{2}$, the third quark must be of different flavour to satisfy Pauli's Exclusion Principle.
	
	{
		\color{red}An OK argument, but key point's antisymmetry of composite wavefunction under particle exchange.
		\begin{equation*}
			\begin{matrix}
				\Qdown\Qdown\Qup \\
				\uparrow\uparrow\downarrow \textnormal{ not possible} \\
				\left.
				\begin{matrix}
					\uparrow\downarrow\uparrow \\
					\downarrow\uparrow\uparrow
				\end{matrix}
				\right\} \mtext{singlet so one particle state only}
			\end{matrix}
			\qquad
			\begin{matrix}
				\Qdown\Qdown\Qdown \\
				\left.
				\begin{matrix}
					\uparrow\uparrow\downarrow \\
					\uparrow\downarrow\uparrow \\
					\downarrow\uparrow\uparrow
				\end{matrix}
				\right\} \mtext{all impossible!}
			\end{matrix}
		\end{equation*}
		Antisymmetry w.r.t. exchange.
	}
	
	\part Again, since baryons are fermions, the parity of its wavefunction must be antisymmetric under particle exchange.
	
	The centre of the Eightfold Way contains $\Qup$, $\Qdown$ and $\Qstrange$.
	There exists 2 possible ways of arranging the flavour wavefunction for it to be antisymmetric:
	\begin{align*}
		\Qup\Qdown\Qstrange &+ \Qdown\Qstrange\Qup &- \Qstrange\Qup\Qdown &- \Qdown\Qup\Qstrange &- \Qup\Qstrange\Qdown &- \Qstrange\Qdown\Qup &\qquad \textnormal{about} &\qquad (12) \leftrightarrow 3 \\
		\Qup\Qdown\Qstrange &+ \Qdown\Qstrange\Qup &- \Qstrange\Qup\Qdown &- \Qdown\Qup\Qstrange &- \Qup\Qstrange\Qdown &- \Qstrange\Qdown\Qup &\qquad \textnormal{about} &\qquad 1 \leftrightarrow (23)
	\end{align*}
	
	And those wavefunctions correspond to the particles $\Sigmaz$ and $\Lambdaz$.
	{\color{red}Check \url{https://physics.stackexchange.com/a/321439}}
	
	Spin wavefunction:
	\begin{tabular}{c c}
		& $\Qdown\Qup\Qstrange$ \\
		$\Sigmaz$ & $\uparrow\uparrow\downarrow$ \\
		\multirow{2}{*}{$\Lambdaz$} & $\uparrow\downarrow\uparrow$ \\
		& $\downarrow\uparrow\uparrow$
	\end{tabular}
	where $\Lambdaz$ has the isospin degeneracy (approximately symmetric) so identical $\Psi$ $\Rightarrow$ some mass difference but very small so indistinguishable.
	
	\part The mass and charge of the particles should indicate that the combination of 3 quarks / antiquarks would explain these properties, e.g. to make up a $+1$ charge, there can be $\Qup\bar{\Qdown}$ or $\Qup\Qup\Qdown$, but the large mass of proton would suggest that $\Qup\Qup\Qdown$ would be its composition instead.
	Also the fact that $J=\dfrac{1}{2}$ would indicate that the number of quarks must be odd.
	
	\part For $\Sigmaz / \Lambdaz$:
	\begin{equation}
		S+B=0, \qquad I_z=0
		\label{eqn:q4-sigma-lambda}
	\end{equation}
	
	For $\Xiz$:
	\begin{equation}
		S+B=-1, \qquad I_z=+\frac{1}{2}
		\label{eqn:q4-xi}
	\end{equation}
	
	So try:
	\begin{align*}
		(S+B) &= mI_z + c \\
		\Rightarrow c &= 0 \mtext{from \eqref{eqn:q4-sigma-lambda}}
	\end{align*}
	
	From \eqref{eqn:q4-xi},
	\begin{align*}
		-1 &= \frac{1}{2}m \\
		\Rightarrow m &= -2 \\
		\textnormal{Gell-Mann Nishijima relation for $Q=0$: } (S+B) = -2I_z
	\end{align*}
	
	Similarly for $Q=+1$, $Y(p)$:
	\begin{equation*}
		S+B=+1, \qquad I_z=+\frac{1}{2}
	\end{equation*}
	
	$\Sigmap$:
	\begin{equation*}
		S+B=0, \qquad I_z=1
	\end{equation*}
	
	So $(S+B) = -2I_z + 2$.
	
	Hence the full relation is:
	\begin{align*}
		(S+B) &= -2I_z + 2Q \\
		\Rightarrow Q &= \frac{S+B}{2} + I_z
	\end{align*}
	
	For $\Qup$, $I_z=+\dfrac{1}{2}$, $B=\dfrac{1}{3}$, $S=0$:
	\begin{equation*}
		Q = \frac{1}{6}+ \frac{1}{2} = +\frac{2}{3}
	\end{equation*}
	
	For $\Qdown$, $I_z=-\dfrac{1}{2}$, $B=\dfrac{1}{3}$, $S=0$:
	\begin{equation*}
		Q = \frac{1}{6} - \frac{1}{2} = -\frac{1}{3}
	\end{equation*}
	
	For $\Qstrange$, $I_z=0$, $B=\dfrac{1}{3}$, $S=1$:
	\begin{equation*}
		Q = -\frac{2}{6} = -\frac{1}{3}
	\end{equation*}
	
	{
		\color{red}Can I get this directly directly from assuming quark properties?
		
		Better use ansatz $Q=\alpha(I_z) + \beta(S+B)$
		
		$Q=+1$: p=uud
		
		$Q=0$: n=dud
		
		So $u-d=1$ $\Rightarrow$ repeat and solve
	}
	
	\part Strong interaction conserves quark number $\Rightarrow$ new particle must be meson.
	{\color{red}How do I know if it's strong?}
	
	Also conservation of charge gives $Q=0$ for the new particle.
	
	Thus the particle must be $\piz$.
	
	\image{.5\linewidth}{q4-momentum}
	Conservation of energy gives:
	\begin{align*}
		m_{\Sigmaz} c^2 &= m_{\Lambdaz} c^2 + E_{\piz} \mtext{since $m_{\Lambdaz} \gg m_{\piz}$} \\
		E_{\piz} &= \rbracket{m_{\Sigmaz} - m_{\Lambdaz}} c^2 \\
		&= \SI{77}{\mega\electronvolt}
	\end{align*}
	{\color{red}Not the energy, but a good estimate of rest mass.}
	
	But $m_{\piz} c^2 > E_{\piz}$!
	So the resulting particle has to be photon instead (preserves quark content too).
	$\Sigmaz \to \neutron \piz$ and $\Sigmaz \to \proton \pin$ not observed since quark content is not conserved.
	
	\part Propose a basis representation as follows:
	\image{.7\linewidth}{q4-basis}
	Then combining the quarks to form a $J=0$ meson nonet:
	\image{.7\linewidth}{q4-meson-nonet}
	Again, mesons are also fermions therefore there are three ways of arranging antisymmetric flavour wavefunctions:
	\begin{equation*}
		\Qup\bar{\Qup} - \bar{\Qup}\Qup,\qquad \Qdown\bar{\Qdown} - \bar{\Qdown}\Qdown,\qquad \Qstrange\bar{\Qstrange} - \bar{\Qstrange}\Qstrange
	\end{equation*}
	
	Mass difference: $\Qup\bar{\Qup}$, $\Qdown\bar{\Qdown}$ have approximate symmetry so they occupy the same $E$ level.
	But $\Qstrange\bar{\Qstrange}$ has significantly larger mass and difference from $\Qup$ and $\Qdown$.
\end{parts}