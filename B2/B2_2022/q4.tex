\draft
\begin{parts}
	\part Three-force:
	\begin{align*}
		\mathbf{F} = \deri{\mathbf{p}}{t} \Rightarrow \mathbf{p}(t) &= \mathbf{F}t \\
		\gamma mv &= \mathbf{F}t \\
		\Rightarrow \frac{mv}{\sqrt{1-\frac{v^2}{c^2}}} &= Ft \\
		m^2 v^2 &= F^2 t^2 \rbracket{1-\frac{v^2}{c^2}} \\
		\rbracket{m^2 + \frac{F^2 t^2}{c^2}} v^2 &= F^2 t^2 \\
		v &= \frac{Ft}{\sqrt{m^2 + \frac{F^2 t^2}{c^2}}}
	\end{align*}
	Note that as $t\to\infty$, $v\to\frac{Ft}{\frac{Ft}{c}} = c$.
	
	\part With 4-momentum $\mathsf{P}^\mu = (E/c, \mathbf{p})$, we have the 4-force:
	\begin{align*}
		\mathsf{F}^\mu = \deri{\mathsf{P}^\mu}{\tau} &= \gamma\deri{\mathsf{P}^\mu}{t} \\
		&= \rbracket{\frac{\gamma W}{c}, \gamma\mathbf{f}}
	\end{align*}
	where $W=\deri{E}{t}$, $\mathbf{f}=\deri{\mathbf{p}}{t}$.
	
	We also need the derivative of $\gamma$ w.r.t. $t$:
	\begin{align*}
		\deri{\gamma}{t} &= -\frac{1}{2}\rbracket{1-\frac{v^2}{c^2}}^{-3/2} \cdot \rbracket{-\frac{2v\cdot a}{c^2}} \\
		&= \gamma^3 \frac{v\cdot a}{c^2}
	\end{align*}
	
	From energy $E=\gamma mc^2$,
	\begin{align*}
		\deri{E}{t} &= \dot{\gamma}mc^2 \\
		&= \gamma^3 mc^2 \frac{\mathbf{v}\cdot\mathbf{a}}{c^2} = \gamma \frac{\mathbf{F}\cdot\mathbf{v}}{c}
	\end{align*}
	
	We also have:
	\begin{align*}
		\gamma\mathbf{f} = \deri{\mathbf{p}}{t} &= \dot{\gamma}m\mathbf{v} + \gamma m\mathbf{a} \\
		\Rightarrow \gamma^3 m \frac{\mathbf{v}\cdot\mathbf{a}}{c^2} \mathbf{v} + \gamma m\mathbf{a} \\
		&= m\gamma^3 \rbracket{\frac{\mathbf{v}\cdot\mathbf{a}}{c^2} \mathbf{v} + \rbracket{1-\frac{v^2}{c^2}} \mathbf{a}} \\
		&= m\gamma^3 \rbracket{\mathbf{a} + \frac{-v^2 \mathbf{a} + \rbracket{\mathbf{v}\cdot\mathbf{a}}\mathbf{v}}{c^2}} \\
		&= m\gamma^3 \rbracket{\mathbf{a} + \frac{\mathbf{v}\times\mathbf{v}\times\mathbf{a}}{c^2}}
	\end{align*}
	
	\part Larmor's formula:
	\begin{equation*}
		P = -\frac{1}{6\pi\permittivity} \frac{q^2}{m^2 c^3} \mathsf{F}^\alpha \mathsf{F}_\alpha
	\end{equation*}
	Note that only the contraction plays a critical role in differentiating the power loss in both cases.
	
	\image{.2\linewidth}{q4-linac}
	For the linac in LAB frame:
	\begin{align*}
		\mathsf{F}^\alpha \mathsf{F}_\alpha &= -\frac{\gamma^2 \rbracket{\mathbf{F}\cdot\mathbf{v}}^2}{c^2} + \gamma^2 F^2 \\
		&= \gamma^2 F^2 \underbracket{\rbracket{1-\frac{v^2}{c^2}}}_{\gamma^{-2}} \mtext{since $\mathbf{F}\parallel\mathbf{v}$} \\
		&= F^2 \\
		&= \gamma^4 m^2 a^2
	\end{align*}
	
	\image{.2\linewidth}{q4-synchrotron}
	Similarly, for a synchrotron in LAB frame:
	\begin{align*}
		\mathsf{F}^\alpha \mathsf{F}_\alpha &= -\frac{\gamma^2 \cancelto{0}{\rbracket{\mathbf{F}\cdot\mathbf{v}}^2}}{c^2} + \gamma^2 F^2 \\
		&= \gamma^2 F^2 \\
		&= \gamma^2 m^2 a^2
	\end{align*}
	where the difference arises from the fact that $\mathbf{F}\perp\mathbf{v}$ in a synchroton.
	
	Thus for the same energy (i.e. same $\gamma$) and the same acceleration, a linac configuration would have higher power dissipation than a synchrotron.
	
	\part Energy of a particle $E=\gamma mc^2$.
	So with larger mass, one can pack more $E$ without increasing $\gamma$.
	Hence muons are the better choice, provided that their decays are accounted for.
	
	Additionally, since the synchroton has lower power dissipation as shown above, using it would be more economical.
\end{parts}