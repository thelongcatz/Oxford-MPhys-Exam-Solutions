\draft
\begin{parts}
	\part For each $\electron$, it has energy of form $\hat{H}_i = \hat{\mathbf{p}}^2/2m - Ze^2/4\pi\permittivity \hat{r}_i$, where the first term is non-relativistic kinetic energy, the second being the Coulomb potential of the nucleus with atomic number $Z$.
	
	Expressing $\hat{\mathbf{p}^2}$ in position representation gives $-\hbar^2 \nabla^2$.
	
	The final part, $e^2/4\pi\permittivity \hat{r}_{12}$, of the overal Hamiltonian is the inter-$\electron$ repulsion.
	
	For 2 identical particles, since there is no way of distinguishing one from another, we have exchange symmetry in the system -- that is a wavefunction must be an eigenfunction of the parity operator.
	
	Parity operator is defined such that $\hat{\mathcal{P}}^2 \psi = \psi$ $\Rightarrow$ $\hat{\mathcal{P}} \psi = \pm\psi$ and so the wavefunction must be either symmetric or antisymmetric.
	
	For the $1s^2$ config, Pauli's exclusion principle tells us that two $\electron$, as fermions, could not occupy the same quantum stute, hence the wavefunction must be antisymmetric so that $\psi \to 0$ as the $\electron$ draws close.
	
	For the $1s2p$ config, since the $\electron$ occupy different orbitals, Pauli's exclusion principle does not apply and thus the wavefunction may be either symmetric or antisymmetric.
	
	The nature of fermions being antisymmetric also mandates that the spin wavefunction must be symmetric ($S=1$) for antisymmetric spatial $\psi$, antisymmetric ($S=0$) for symmetric spatial $\psi$.
	
	Energy diagram of $1s^2$ and $1s2p$ config of He:
	\image{.7\linewidth}{q1-energy-level}
	
	\part \todo As $Z$ increases, the expectation value of $1/r_{12}$ should decrease as the system increasingly pulls the $\electron$ more, making $\electron$ at different orbitals more far away.
	
	Modelling IE as $aZ^2 + b$:
	\begin{align*}
		\Rightarrow a &= \SI{11.89}{\electronvolt} \\
		b &= \SI{-21.75}{\electronvolt}
	\end{align*}
	So Al${}^{11+}$ should have IE $\SI{1952}{\electronvolt}$, Al${}^{12+}$ should have IE given by Rydberg level:
	\begin{align*}
		E &= \frac{hcR_\infty}{1^2} \cdot Z^2 \\
		&= \SI{2300}{\electronvolt}
	\end{align*}
	
	From g to a,
	\begin{tabular}{c c c}
		g: Al${}^{12+}$ & d: Al${}^{11+}$ & a: \\
		f: Al${}^{11+}$ & c: & \\
		e: Al & b: &
	\end{tabular}
	
	h is \textit{bremsstrahlung}?
\end{parts}