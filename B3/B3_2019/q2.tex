\draft
\begin{parts}
	\part The emission of X-rays is due to the introduction of holes in the inner shell via $\electron$ bombardment, which knocks off the inner $\electron$.
	In addition, X-rays may also be emitted via bremsstrahlung when the incoming $\electron$ accelerates under the atomic potential.
	
	\image{.8\linewidth}{q2-x-ray-probe}
	The emission spectrum is a curve with several resonance peaks due to the background bremsstrahlung, while the peaks are simply the emission of X-rays from the hole transitioning away from the inner shells.
	
	Furthermore, the peaks will not appear until the probed energy ($E_e$) exceeds the energy level of the inner shells (see dashed line above).
	And the cut off point of the bremsstrahlung curve is simply the energy of the $\electron$ beam.
	
	\part The absorption and emission spectra will have the same peaks as they are inherent to the atomic structure.
	
	However, the absorption spectrum lacks the bremsstrahlung features as a photon cannot emit another under EM field.
	
	\image{.8\linewidth}{q2-emission-absorption}
	
	\part Characteristic X-rays are emissions due to the introduction of holes in the inner shells.
	The final set of lines is due to the innermost ($K$) shell emission.
	
	The transitions can be $K \to L$, $K \to M \ldots$
	\image{.3\linewidth}{q2-char-x-ray}
	
	\part The emission of $\electron$ is due to the Auger effect where energy of emission goes to ejecting $\electron$ instead of emitting a photon.
	\image{.5\linewidth}{q2-auger}
	So electron energy:
	\begin{gather*}
		T = E_\textnormal{probe} + E \\
		\Rightarrow -E = E_\textnormal{probe} - T \\
		\Rightarrow \SI{7.721}{\kilo\electronvolt}, \SI{7.426}{\kilo\electronvolt}, \SI{7.256}{\kilo\electronvolt}, \SI{7.183}{\kilo\electronvolt}, \SI{7.002}{\kilo\electronvolt}, \SI{6.833}{\kilo\electronvolt}
	\end{gather*}
	\image{.8\linewidth}{q2-auger-transitions}
	
	\part We are having \ion{Ag}{++} instead of \ion{Ag}{+} so energy level not exactly the same.
	Conservation of momentum so measured lower than theoretical.
	\image{.2\linewidth}{q2-ag}
\end{parts}