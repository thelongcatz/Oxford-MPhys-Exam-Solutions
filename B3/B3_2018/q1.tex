\draft
\begin{parts}
	\part As the $\electron$ orbits around the nucleus with angular momentum $\mathbf{l} \neq 0$, it experiences magnetic field in its rest trame due to the transformation of the electric field from the nucleus.
	The interaction between this magnefic field and the intrinsic magnetic moment of the $\electron$ leads to a splitting in energy level called fine structure.
	
	\image{.2\linewidth}{q1-electron-orbit}
	In $\electron$ rest frame,
	\begin{align*}
		\mathbf{B} &= -\frac{\mathbf{v}\times\mathbf{B}}{c^2} \mtext{and} \\
		\mathbf{v} &= \frac{\mathbf{p}}{m},\, -e\mathbf{E} = -\pderi{U}{r}
	\end{align*}
	where $U(r)$ is Coulomb potential of the nucleus.
	
	\begin{align*}
		\Rightarrow \mathbf{B} &= -\frac{1}{m_e c^2} \rbracket{\frac{1}{er} \pderi{U}{r}} \mathbf{p}\times\mathbf{r} \mtext{since } \hat{\mathbf{r}}=\frac{\mathbf{r}}{r} \\
		&= \frac{\hbar}{m_e c^2} \rbracket{\frac{1}{er} \pderi{U}{r}} \mathbf{l} \mtext{for } \mathbf{r}\times\mathbf{p}=\hbar\mathbf{l}
	\end{align*}
	
	Also magnetic moment of $\electron$:
	\begin{equation*}
		\bm{\mu}_s = -g_s \bohrmagneton \mathbf{s}
	\end{equation*}
	with $g_s \simeq 2$ the gyromagnetic ratio, $\bohrmagneton = \dfrac{e\hbar}{2m_e}$.
	
	Hamiltonian of magnetic interaction:
	\begin{align*}
		\Delta\hat{H}_\textnormal{SO} &= -\bm{\mu}_s \cdot \mathbf{B} \\
		&= g_s \bohrmagneton \cdot \frac{\hbar}{m_e c^2} \rbracket{\frac{1}{er} \pderi{U}{r}} \mathbf{s}\cdot\mathbf{l} \\
		&\rightarrow (g_s - 1) \frac{e\hbar}{2m_e} \cdot \frac{\hbar}{m_e c^2} \rbracket{\frac{1}{er} \frac{Ze^2}{4\pi\permittivity r^2}} \mathbf{s}\cdot\mathbf{l}
	\end{align*}
	by Thomas precession and $\pdiagderi{U}{r} = \pdiagdiff{r}\rbracket{-\dfrac{Ze^2}{4\pi\permittivity r}}$.
	
	\begin{align*}
		\Delta E_\textnormal{SO} &= \frac{\hbar^2}{2m_e^2 c^2} \avg{\frac{1}{r^3}} \frac{Ze^2}{4\pi\permittivity} \underbracket{\avg{\mathbf{s}\cdot\mathbf{l}}}_{\mathclap{\frac{1}{2}\avg{\hat{\mathbf{j}}^2 - \hat{\mathbf{l}}^2 - \hat{\mathbf{s}}^2}}} \\
		&= \frac{\beta}{2} \sbracket{j(j+1)-l(l+1)-s(s+1)}
	\end{align*}
	where
	\begin{align*}
		\beta &= \frac{\hbar}{2m_e^2 c^2} \avg{\frac{1}{r^3}} \cdot \frac{Ze^2}{4\pi\permittivity} \\
		&= \frac{\hbar}{2m_e^2 c^2} \cdot \rbracket{\frac{Z}{na_0}}^3 \cdot \frac{1}{l(l+\frac{1}{2})(l+1)} \cdot \frac{Ze^2}{4\pi\permittivity} \\
		&= \frac{\hbar}{2m_e^2 c^2} \cdot \frac{e^2}{4\pi\permittivity} \cdot \frac{Z^4}{l(l+\frac{1}{2})(l+1)(na_0)^3}
	\end{align*}
	
	\part $n=2$ shells: $2s$, $2p$
	
	$n=3$ shells: $3s$, $3p$, $3d$
	
	Config $2s$ $\to$ Term ${}^2 S$ $\to$ Level ${}^2 S_{1/2}$
	
	Config $2p$ $\to$ Term ${}^2 P$ $\to$ Level ${}^2 P_{1/2}$, ${}^2 P_{3/2}$
	
	Config $3s$ $\to$ Term ${}^2 S$ $\to$ Level ${}^2 S_{1/2}$
	
	Config $3p$ $\to$ Term ${}^2 P$ $\to$ Level ${}^2 P_{1/2}$, ${}^2 P_{3/2}$
	
	Config $3d$ $\to$ Term ${}^2 D$ $\to$ Level ${}^2 D_{3/2}$, ${}^2 D_{5/2}$
	
	Energy shifts:
	\begin{align*}
		2p \; {}^2 P_{1/2}: \; \beta &= \frac{\hbar^2}{2m_e^2c^2} \cdot \alpha 2\pi hc \cdot \frac{2^4}{1\rbracket{\frac{3}{2}} (2) (2a_0)^3} \\
		&= \frac{(\SI{197.33}{\mega\electronvolt\femto\metre})^3 \cdot 16\pi}{(\SI{0.511}{\mega\electronvolt})^2 \cdot 24(\SI{5.292e4}{\femto\metre})^3} \\
		&= \SI{4.158e-7}{\mega\electronvolt} = \SI{0.4158}{\electronvolt} \\
		\Rightarrow \Delta E_\textnormal{SO}\rbracket{j=\frac{1}{2}} &= \frac{\beta}{2} \sbracket{\frac{1}{2}\rbracket{\frac{3}{2}} - 1(2) - \frac{1}{2}\rbracket{\frac{3}{2}}} \\
		&= -\beta =\SI{-0.4158}{\electronvolt}
	\end{align*}
	
	\begin{align*}
		2p \; {}^2 P_{3/2}: \; \Delta E_\textnormal{SO} &= \frac{\beta}{2} \sbracket{\frac{3}{2}\rbracket{\frac{5}{2}} - 1(2) - \frac{1}{2\rbracket{\frac{3}{2}}}} \\
		&= \frac{\beta}{2} = \SI{0.2079}{\electronvolt}
	\end{align*}
	
	\begin{align*}
		3p \; {}^2 P_{1/2}: \; \beta &= \frac{(\SI{197.33}{\mega\electronvolt\femto\metre})^3 \cdot 2^4 \pi}{(\SI{0.511}{\mega\electronvolt})^2 \cdot 3 \cdot 3^3 (\SI{5.292e4}{\femto\metre})^3} \mtext{similarly} \\
		&= \SI{1.232e-7}{\mega\electronvolt} = \SI{0.1232}{\electronvolt} \\
		\Delta E_\textnormal{SO}\rbracket{j=\frac{1}{2}} &= \frac{\beta}{2} \cdot \sbracket{\frac{1}{2}\rbracket{\frac{3}{2}}\ldots} \\
		&= -\beta \\
		&= \SI{-0.1232}{\electronvolt} \mtext{similarly} \\[1em]
		{}^2 P_{3/2}: \; \Delta E\rbracket{j=\frac{3}{2}} &= \frac{\beta}{2} \\
		&= \SI{0.0616}{\electronvolt}
	\end{align*}
	
	\begin{align*}
		3d \; {}^2 D_{3/2}: \; \beta &= \frac{(\SI{197.33}{\mega\electronvolt\femto\metre})^3 \cdot 16 \pi}{(\SI{0.511}{\mega\electronvolt})^2 \cdot 2 \rbracket{\frac{5}{2}} (3) \cdot 3^3 (\SI{5.292e4}{\femto\metre})^3} \\
		&= \SI{2.464e-8}{\mega\electronvolt} \\
		&= \SI{0.02464}{\electronvolt} \\
		\Delta E\rbracket{j=\frac{3}{2}} &= \frac{\beta}{2} \sbracket{\frac{3}{2}\rbracket{\frac{5}{2}} - 2(3) - \frac{1}{2}\rbracket{\frac{3}{2}}} \\
		&= -\frac{3\beta}{2} \\
		&= \SI{-0.03696}{\electronvolt} \\[1em]
		\Delta E\rbracket{j=\frac{5}{2}} &= \frac{\beta}{2} \sbracket{\frac{5}{2}\rbracket{\frac{9}{2}} - 2(3) - \frac{1}{2}\rbracket{\frac{3}{2}}} \\
		&= \beta \\
		&= \SI{0.02464}{\electronvolt}
	\end{align*}
	
	\newpage
	Energy level diagram:
	\image{.8\linewidth}{q1-energy-level}
	
	Electric dipole selection rule:
	
	\begin{tabular}{p{.11\linewidth} p{.22\linewidth} p{.22\linewidth} p{.3\linewidth}}
		$\Delta n=$ any & $\Delta l=\pm 1$ & & \\
		$\Delta S=0$ & $\Delta L=0, \pm 1$ ($0 \nrightarrow 0$) & $\Delta J=0, \pm 1 (0 \nrightarrow 0)$ & $\Delta M_J = 0, \pm 1$ ($0 \nrightarrow 0$ iff $\Delta J=0$)
	\end{tabular}
	
	So possible transitions:
	
	\begin{tikzpicture}[auto]
		\matrix (2s) [matrix of math nodes,row sep=1cm,column sep=16mm] {
			2s/3s \;\; {}^2 S_{1/2} & 2p \;\; {}^2 P_{1/2} \\
			& 2p \;\; {}^2 P_{3/2} \\
			& 3p \;\; {}^2 P_{1/2} \\
			& 3p \;\; {}^2 P_{3/2} \\
		};
		\draw (2s-1-1)--(2s-1-2);
		\draw (2s-1-1)--(2s-2-2);
		\draw (2s-1-1)--(2s-3-2);
		\draw (2s-1-1)--(2s-4-2);
	\end{tikzpicture}
	
	\part $\Omega^-$ baryon belongs to the $J^P = \dfrac{3}{2}^+$ group so it possesses $s=\dfrac{3}{2}$.
	
	For the exotic hydrogenic ion \ion{Pb}{81+}, replace $m_e \to m_\Omega$ and $s=\dfrac{1}{2} \to \dfrac{3}{2}$.
	
	So:
	\begin{align*}
		\beta &= \frac{\hbar^2 c^2}{2m_\Omega^2 c^4} \cdot 2\pi\alpha\hbar c \cdot \frac{82^4}{9\rbracket{\frac{19}{2}}(10)(10)^3 (a_0)^3} \\
		&= \frac{(\SI{197.33}{\mega\electronvolt\femto\metre})^3 \cdot 82^4 \pi}{(\SI{1672}{\mega\electronvolt})^2 \cdot \num{855000} (\SI{5.292e4}{\femto\metre})^3} \\
		&= \SI{3.081e-12}{\mega\electronvolt} \\
		&= \SI{3.081e-6}{\electronvolt}
	\end{align*}
	
	Config $10l$ $\to$ Term ${}^4 L$ $\to$ Level ${}^4 L_{15/2}$, ${}^4 L_{17/2}$, ${}^4 L_{19/2}$, ${}^4 L_{21/2}$
	
	So:
	\begin{align*}
		\Delta E\rbracket{j=\frac{15}{2}} &= \frac{\beta}{2} \sbracket{\frac{15}{2}\rbracket{\frac{17}{2}} - 9(10) - \frac{3}{2}\rbracket{\frac{5}{2}}} \\
		&= -15\beta = \SI{-4.621e-5}{\electronvolt} \\[1em]
		\Delta E\rbracket{j=\frac{17}{2}} &= \frac{\beta}{2} \sbracket{\frac{17}{2}\rbracket{\frac{19}{2}} - 9(10) - \frac{3}{2}\rbracket{\frac{5}{2}}} \\
		&= -\frac{13}{2}\beta = \SI{-2.003e-5}{\electronvolt} \\[1em]
		\Delta E\rbracket{j=\frac{19}{2}} &= \frac{\beta}{2} \sbracket{\frac{19}{2}\rbracket{\frac{21}{2}} - 9(10) - \frac{3}{2}\rbracket{\frac{5}{2}}} \\
		&= 3\beta = \SI{9.243e-6}{\electronvolt} \\[1em]
		\Delta E\rbracket{j=\frac{21}{2}} &= \frac{\beta}{2} \sbracket{\frac{21}{2}\rbracket{\frac{23}{2}} - 9(10) - \frac{3}{2}\rbracket{\frac{5}{2}}} \\
		&= \frac{27}{2}\beta = \SI{4.159e-5}{\electronvolt}
	\end{align*}
	\image{.7\linewidth}{q1-energy-diff}
	
	\part Possible spins:
	\begin{align*}
		S &= \abs{\frac{3}{2}-\frac{3}{2}} \ldots \rbracket{\frac{3}{2}+\frac{3}{2}} \\
		&= 0, 1, 2, 3
	\end{align*}
	
	For excited state $1s2s$ $\to$
	\begin{tabular}{l c c c c}
		Term & ${}^1 S$ & ${}^3 S$ & ${}^5 S$, ${}^7 S$ \\
		& \vline & \vline & \vline & \vline \\
		Level & ${}^1 S_0$ & ${}^3 S_1$ & ${}^5 S_2$ & ${}^7 S_3$
	\end{tabular}
	
	For $1s2p$,
	
	\begin{tabular}{l c c c c}
		Term & ${}^1 P$ & ${}^3 P$ & ${}^5 P$ & ${}^7 P$ \\
		&& $\downarrow$ && \\
		Level & ${}^1 P_1$ & ${}^3 P_0$, ${}^3 P_1$, ${}^3 P_2$ & ${}^5 P_1$, ${}^5 P_2$, ${}^5 P_3$ & ${}^7 P_2$, ${}^7 P_3$, ${}^7 P_4$
	\end{tabular}
	
	For ground state $1s^2$, Pauli exclusion principle states that no 2 fermions may occupy the same state, so the 2 $\Omega^-$'s must not have aligned spins.
	So the available spins are $S=0, 1, 2$.
	
	$\Rightarrow$
	\begin{tabular}{l c c c}
		Ground term & ${}^1 S$ & ${}^3 S$ & ${}^5 S$ \\
		Level & ${}^1 S_0$ & $\bcancel{\cancel{{}^3 S_1}}$ & ${}^5 S_2$
	\end{tabular}
	
	This differs from helium whereby only ${}^1 S_0$ is available as $\electron$ has spin $\dfrac{1}{2}$ instead of $\dfrac{3}{2}$.
\end{parts}