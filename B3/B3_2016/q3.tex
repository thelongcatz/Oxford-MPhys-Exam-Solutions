\draft
\begin{parts}
	\part A 3-level laser has its metastable level close to the ground level, making them essentially the same state.
	A 4-level laser, however, has a distinguishable metastable term at typical temperature range.
	
	Example of 3-level laser: He-Ne
	
	Example of 4-level laser: Nd:YAG
	
	Both lasers operate by pumping the ground state population to an unstable upper state, which then immediately decays into the upper lasing state.
	After lasing the decayed population decays spontaneously back to the ground state.
	
	The pump rate of 3-level laser is proportional to the difference in population if both ground and unstable states.
	At thermal equilibrium, the population will be roughly the same, making pumping difficult.
	In contrast, 4-level lasers do not suffer from this and thus may operate at lower power continuously.
	
	\image{.4\linewidth}{q3-population-dynamics}
	In the absence of radiation, the rate equations read:
	\begin{align*}
		\deri{N_2}{t} &= R_2 - N_2 A_{21} \\
		\deri{N_1}{t} &= R_1 + N_2 A_{21} - \frac{N_1}{\tau_1}
	\end{align*}
	
	At thermal equilibrium,
	\begin{align*}
		\deri{N_2}{t} = \deri{N_1}{t} &= 0 \\
		\Rightarrow N_2 &= \frac{R_2}{A_{21}} \\
		N_1 &= R_1 \tau_1 + R_2 \tau_1
	\end{align*}
	\begin{align*}
		N^* &= N_2 - \frac{g_2}{g_1}N_1 \\
		&= \frac{R_2}{A_{21}} - \frac{g_2}{g_1}\rbracket{R_1 \tau_1 + R_2 \tau_1} \\
		&= R_2\sbracket{\frac{1}{A_{21}} - \frac{g_2}{g_1}\tau_1}
	\end{align*}
	for $R_1 = 0$.
	
	Now in the presence of radiation:
	\begin{align*}
		\deri{N_2}{t} &= R_2 + \rho B_{12}N_1 - \rho B_{21}N_2 - A_{21}N_2 \\
		\deri{N_1}{t} &= R_1 - \rho B_{12}N_1 + \rho B_{21}N_2 + A_{21}N_2 - \frac{N_1}{\tau_1}
	\end{align*}
	where $\rho$ is spectral energy density, $B$ is Einstein's B coefficient.
	
	Change in number of photons in a volume element $A\inftsml{z}$ (assuming narrowband so $\rho\to\rho g_H \delta\omega$ where $g_H(\omega - \omega_0)$ is homogeneous broadening function, $\delta\omega$ is the width of the input frequency):
	\begin{equation*}
		\Delta N_\gamma = \sbracket{\rho B_{21}N_2 - \rho B_{12}N_1} g_H \delta\omega A \inftsml{z}
	\end{equation*}
	
	Energy change to the beam:
	\begin{align*}
		\inftsml{I} \bcancel{A\delta\omega} &= \Delta N_\gamma\cdot\omega_L = \sbracket{\rho B_{21}N_2 - \rho B_{12}N_1} g_H \bcancel{\delta\omega A} \inftsml{z}\cdot\hbar\omega_L \\
		\deri{I}{z} &= \underbracket{\sbracket{N_2 - \frac{g_2}{g_1}N_1}}_{N^*} \underbracket{B_{21}g_H\frac{\hbar\omega_L}{c}}_{\sigma_{21}} I
	\end{align*}
	where $\omega_L$ is the lasing frequency, $\rho=I/c$ and $B_{12}g_1 = g_2 B_{21}$.
	
	Now rewrite the rate equations with $N^*$, $\sigma_{21}$ and $I$:
	\begin{align*}
		\deri{N_2}{t} &= R_2 - N^* \sigma_{21} \frac{I}{\hbar\omega_L} - A_{21}N_2 \\
		\deri{N_1}{t} &= R_1 + N^* \sigma_{21} \frac{I}{\hbar\omega_L} + A_{21}N_2 - \frac{N_1}{\tau_1}
	\end{align*}
	
	At thermal equilibrium,
	\begin{align*}
		N_2 &= \frac{R_2}{A_{21}} - \frac{N^* \sigma_{21} I}{A_{21}\hbar\omega_L} \\
		N_1 &= R_1 \tau_1 + N^* \sigma_{21} \frac{I}{\hbar\omega_L} \tau_1 + A_{21} N_2 \tau_1
	\end{align*}
	
	Substitute $N_2$, $N_1$ into $N^*$:
	\begin{align*}
		N^* &= \rbracket{1-\frac{g_2}{g_1}\tau_1 A_{21}} \sbracket{\frac{R_2}{A_{21}} - N^* \sigma_{21} \frac{I}{\hbar\omega_L} \frac{1}{A_{21}}} - R_1 \tau_1 - N^* \sigma_{21} \frac{I}{\hbar\omega_L} \tau_1 \\
		\Rightarrow N^*(I) &= \frac{N^*(0)}{1 + \cfrac{I}{I_s}}
	\end{align*}
	where
	\begin{align*}
		N^*(0) &= \sbracket{1-\frac{g_2}{g_1}\tau_1 A_{21}} \cdot \frac{R_2}{A_{21}} - R_1 \tau_1 \\
		&= R_2 \rbracket{\frac{1}{A_{21}} - \frac{g_2}{g_1}\tau_1} - R_1 \tau_1
	\end{align*}
	and
	\begin{align*}
		I_s &= \sbracket{\sigma_{21} \frac{1}{\hbar\omega_L}\frac{1}{A_{21}}\rbracket{1-\frac{g_2}{g_1}\tau_1 A_{21}} + \sigma_{21}\frac{1}{\hbar\omega_L}\tau_1}^{-1} \\
		&= \sbracket{\sigma_{21}\frac{1}{\hbar\omega_L}\underbracket{\rbracket{\tau_1 + \frac{1}{A_{21}} + \frac{g_2}{g_1}\tau_1 \frac{1}{A_{21}}}}_{\tau_R}}^{-1} \\
		&= \frac{\hbar\omega_L}{\sigma_{21}\tau_R}
	\end{align*}
	is the saturation intensity.
	
	Beam growth $\diagderi{I}{z} = \alpha(I)I$ where
	\begin{equation*}
		\alpha(I) = N^*(I) \sigma_{21} = \frac{\alpha(0)}{1+\cfrac{I}{I_s}}
	\end{equation*}
	with
	\begin{align*}
		\alpha(0) &= N^*(0) \sigma_{21} \\
		&= R_2 \sigma_{21} \rbracket{\frac{1}{A_{21}}-\frac{g_2}{g_1}\tau_1} - R_1 \sigma_{21} \tau_1
	\end{align*}
	the null intensity gain coefficient.
	
	So:
	\begin{align*}
		\frac{1}{I}+\frac{1}{I_s} \inftsml{I} &= \alpha(0) \inftsml{z} \\
		\ln\rbracket{\frac{I}{I_0}}+\frac{I-I_0}{I_s} &= \alpha(0)z
	\end{align*}
	For $I \gg I_s$, the second term dominates and so $I=I_0 + \alpha(0)I_s z$ and the beam growth is linear as the medium is saturated.
	
	\part For $\sigma_{21} = \SI{4e-15}{\centi\metre\squared}$, $R_2 = \SI{5e19}{\per\second\per\centi\metre\cubed}$, $R_1 = 0$, $A_{21} = \SI{3e6}{\per\second}$, $g_2 = 5$, $g_1 = 3$, $\tau_1 = \SI{2e-8}{\second}$ $\rightarrow$ $\alpha(0) = \num{0.06}$.
	
	Similarly for $A_{21} = \SI{3e6}{\per\second}$, $g_2 = 3$, $\tau_1 = \SI{3e-7}{\second}$, $g_1 = 1$ $\rightarrow$ $\alpha(0) = \num{-0.113}$
\end{parts}