\draft
\begin{parts}
	\part The natural width of a transition is the homogeneous broadening due to the intrinsic lifetime associated with the transition itself.
	This may be illustrated by considering the Uncertainty Principle: $\Delta E \Delta t \ge \hbar/2$
	
	Thus:
	\begin{equation*}
		\Delta E \sim \Delta\omega \sim \frac{1}{\Delta t}
	\end{equation*}
	
	For a gas phase amplifier, Doppler broadening and pressure broadening are the major contributions to the overall broadening -- at higher temperatures the average speed of the atoms increases, increasing both Doppler and pressure broadenings.
	For larger density the pressure broadening will increase as the mean free path decreases.
	
	Rate equations:
	\begin{align*}
		\deri{N_\mathrm{e}}{t} &= \Gamma - \frac{N_\mathrm{e}}{\tau_\mathrm{e}} \\
		\deri{N_m}{t} &= \frac{N_\mathrm{e}}{\tau_\mathrm{e}} - \frac{N_m}{\tau_m} \\
		\deri{N_g}{t} &= \frac{N_m}{\tau_m} - \Gamma
	\end{align*}
	
	At steady state:
	\begin{align*}
		N_\mathrm{e} &= \Gamma\tau_\mathrm{e} \\
		\frac{N_\mathrm{e}}{\tau_\mathrm{e}} &= \frac{N_m}{\tau_m} \\
		N_m &= \Gamma \tau_m
	\end{align*}
	
	So:
	\begin{align*}
		N_\mathrm{e} &= \frac{N_g - N_\mathrm{e}}{\hbar\omega_\textnormal{L}} \sigma_{g\mathrm{e}} I_L \tau_\mathrm{e} \\
		&= \frac{N-N_m-2N_\mathrm{e}}{\hbar\omega_\textnormal{L}} \sigma_{g\mathrm{e}} I_L \tau_\mathrm{e} \\
		&= \frac{N-(\tau_m/\tau_\mathrm{e} + 2)N_\mathrm{e}}{\hbar\omega_\textnormal{L}} \sigma_{g\mathrm{e}} I_L \tau_\mathrm{e}
	\end{align*}
	
	\begin{align*}
		\Rightarrow \rbracket{\hbar\omega_\textnormal{L} + \frac{\tau_m}{\tau_\mathrm{e}} + 2} N_\mathrm{e} &- N \\
		N_\mathrm{e} &= N \sbracket{\frac{\hbar\omega_\textnormal{L}}{\sigma_{g\mathrm{e}} I_L \tau_\mathrm{e}} + \frac{\tau_m}{\tau_\mathrm{e}} + 2}^{-1} \\
		&= N \frac{\sigma_{g\mathrm{e}} I_L \tau_\mathrm{e}}{\hbar\omega_\textnormal{L} + \sigma_{g\mathrm{e}} I_L \tau_m \tau_\mathrm{e} + 2\sigma_{g\mathrm{e}} I_L \tau_m} \\
		&= N \frac{\tau_\mathrm{e}}{\tau_m} \, \frac{\sigma_{g\mathrm{e}} I_L}{\frac{\hbar\omega_\textnormal{L}}{\tau_m} + \sigma_{g\mathrm{e}} I_L (\tau_\mathrm{e} + 2)} \\
		&=  N \frac{\tau_\mathrm{e}}{\tau_m} \, \cfrac{\cfrac{\sigma_{g\mathrm{e}} I_L \tau_m}{\hbar\omega_\textnormal{L}}}{1+\cfrac{\sigma_{g\mathrm{e}} I_L \tau_m}{\hbar\omega_\textnormal{L}} \rbracket{\tau_\mathrm{e} + 2}}
	\end{align*}
	
	Hence $x=\dfrac{\sigma_{g\mathrm{e}} I_L \tau_m}{\hbar\omega_\textnormal{L}}$, $\dfrac{\tau_R}{\tau_m} = \tau_\mathrm{e} + 2$ $\Rightarrow$ $\tau_R = \tau_m (\tau_\mathrm{e} + 2)$.
	And:
	\begin{equation*}
		I_F \propto \frac{N_\mathrm{e}}{\tau_\mathrm{e}} = \frac{N}{\tau_m} \cdot \frac{x}{1+x\rbracket{\cfrac{\tau_R}{\tau_m}}}
	\end{equation*}
	
	For $\tau_m \gg \tau_\mathrm{e}$, $\tau_R \simeq \tau_m$:
	\begin{equation*}
		\frac{N_\mathrm{e}}{\tau_\mathrm{e}} = \frac{N}{\tau_m} \cdot \frac{x}{1+x}
	\end{equation*}
	
	\image{.8\linewidth}{q4-if-x-sketch}
	
	When $x$ is not small, $I_p$ so the broadening is of width $1/\tau_m$ instead.
	In intermediate regime, the broadening is of width $\frac{x}{\tau_m \rbracket{1_x\rbracket{\frac{\tau_R}{\tau_m}}}}$.
\end{parts}