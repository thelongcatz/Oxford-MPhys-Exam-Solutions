\draft
\begin{parts}
	\part Note that the $\electron$ are indistinguishable, therefore it follows an exchange symmetry and thus $\hat{p}^2 \phi = \phi$ where $\hat{p}$ is the parity operator.
	
	The definition of $\hat{p}$ implies that $\hat{p}\phi = \pm\phi$ and thus $\phi$ may be either symmetric or antisymmetric.
	
	Since $\electron$ are fermions, they are antisymmetric under $\hat{p}$, however we may write $\phi=\phi_\textnormal{space} \phi_\textnormal{spin}$ so we have 2 cases:
	\begin{enumerate}
		\item Symmetric $\phi_\textnormal{space}$, antisymmetric $\phi_\textnormal{spin}$; and
		\item Antisymmetric $\phi_\textnormal{space}$, symmetric $\phi_\textnormal{spin}$.
	\end{enumerate}
	
	Therefore a normalised spatial wavefunction should look like this:
	\begin{equation*}
		\phi(\mathbf{r}_1, \mathbf{r}_2) = \frac{1}{\sqrt{2}} \sbracket{u_A(\mathbf{r}_1 u_B(\mathbf{r}_2)) + u_B(\mathbf{r}_1) u_A(\mathbf{r}_2)}
	\end{equation*}
	
	For aligned spin ($S=1$), we have antisymmetric spatial $phi$:
	\begin{align*}
		\bra{\phi}\rbracket{\mathbf{r}_1 - \mathbf{r}_2}^2\ket{\phi} &= \frac{1}{2} \int
		\sbracket{u_A^*(\mathbf{r}_1) u_B^*(\mathbf{r}_2) - u_B^* (\mathbf{r}_1) u_A^* (\mathbf{r}_2)} \\
			&\qquad\qquad (\mathbf{r}_1 - \mathbf{r}_2)^2 \\
			&\qquad\qquad \sbracket{u_A(\mathbf{r}_1) u_B(\mathbf{r}_2) - u_B(\mathbf{r}_1) u_A(\mathbf{r}_2)} \;
			\inftsml{{}^3\mathbf{r}_1} \, \inftsml{{}^3\mathbf{r}_2} \\
		&= \frac{1}{2} \rbracket{2I-2K} \\
		&= I-K
	\end{align*}
	where
	\begin{align*}
		I &= \indefint{\abs{u_A(\mathbf{r}_1)}^2 \abs{u_B(\mathbf{r}_2)}^2 \rbracket{\mathbf{r}_1 - \mathbf{r}_2}^2}{{}^3\mathbf{r}_1}\,\inftsml{{}^3\mathbf{r}_2} \\
		J &= \indefint{u_A^*(\mathbf{r}_1)u_A(\mathbf{r}_2) u_B^*(\mathbf{r}_2)u_B(\mathbf{r}_1) \rbracket{\mathbf{r}_1 - \mathbf{r}_2}}{{}^3\mathbf{r}_1}\,\inftsml{{}^3\mathbf{r}_2}
	\end{align*}
	
	\part Since aligned spins have larger separation, it has lower energy than antialigned spins.
	This means that a configuration is split into different terms with differing $S$.
	
	Since electrical dipole transition forbids change in $S$, this effectively means that there exists groups of lines close together with scale greater than spin-orbit splitting.
	
	However, intercombination lines are possible with non-electric-dipole transitions, e.g. magnetic dipole.
	
	For this ground state, we may invoke Hund's rules:
	\image{.2\linewidth}{q2-lz}
	So the $\electron$ arrange to give net nil orbital angular momentum, $L=0$.
	
	For phosphorous, the Aufbau principle gives config $1s^2 2s^2 2p^6 3s^2 3p^3$ $\Rightarrow$ ground state term ${}^4 S$.
\end{parts}