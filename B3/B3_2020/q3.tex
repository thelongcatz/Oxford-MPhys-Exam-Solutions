\draft
\begin{parts}
	\part \image{.8\linewidth}{q3-population-dynamics}
	When a light passes through the system, it interacts with it by inducing stimulated emission/absorption, which for an incoherent light may be described by Einstein's B coefficient.
	
	Otherwise Rabi oscillations may occur and we may proceed via time-dependent perturbation theory.
	
	Rate equations (for a single frequency):
	\begin{align*}
		\deri{N_2}{t} &= R_2 + B_{12}\rho N_1 - B_{21}\rho N_2 - \frac{N_2}{\tau_2} \\
		\deri{N_1}{t} &= R_1 - B_{12}\rho N_1 + B_{21}\rho N_2 + A_{21}N_2 - \frac{N_1}{\tau_1}
	\end{align*}
	
	\image{.3\linewidth}{q3-gain-dz}
	Energy change within a volume element $A \inftsml{z}$: (replace $\rho\to\rho g\delta\omega$ for finite $\omega$ width)
	\begin{align*}
		\inftsml{\mathcal{I}} \cancel{A} \bcancel{\delta\omega} &= \sbracket{-B_{12}\rho g\bcancel{\delta\omega}N_1 + B_{21}\rho g\bcancel{\delta\omega}N_2} \hbar\omega_0 \cancel{A} \inftsml{z} \\
		\deri{\mathcal{I}}{z} &= \underbracket{\sbracket{N_2 - \frac{g_2}{g_1}N_1}}_{N^*} B_{21}\rho g\hbar\omega_0 \mtext{since } g_1B_{12}=g_2B_{21} \\
		&= N^* \underbracket{B_{21}(\omega - \omega_0)\frac{\hbar\omega_0}{c}}_{\sigma_{21}(\omega - \omega_0)}\mathcal{I}
	\end{align*}
	
	Now rewrite the rate equations with $N^*$, $\sigma_{21}$ and $\defint{0}{\infty}{}{\omega}$ to get $\defint{0}{\infty}{\mathcal{I}}{\omega}$ total intensity:
	\begin{align*}
		\deri{N_2}{t} &= R_2 - N^* \sigma_{21} \frac{I}{\hbar\omega} - \frac{N_2}{\tau_2} \\
		\deri{N_1}{t} &= R_1 + N^* \sigma_{21} \frac{I}{\hbar\omega} + A_{21}N_2 - \frac{N_1}{\tau_1}
	\end{align*}
	
	At steady state,
	\begin{align*}
		N_2 &= R_2\tau_2 - N^* \sigma_{21}\frac{I}{\hbar\omega}\tau_2 \\
		N_1 &= R_1\tau_1 + N^* \sigma_{21}\frac{I}{\hbar\omega}\tau_1 + A_{21}N_2\tau_1
	\end{align*}
	
	Hence:
	\begin{align*}
		N^* &= N_2 - \frac{g_2}{g_1}N_1 \\
		&= R_2\tau_2 - N^* \sigma_{21}\frac{I}{\hbar\omega}\rbracket{\tau_2 - \frac{g_2}{g_1}\tau_1} - \frac{g_2}{g_1}R_1\tau_1 - \frac{g_2}{g_1}A_{21}\tau_1\sbracket{R_2\tau_2 - N^* \sigma_{21}\frac{I}{\hbar\omega}\tau_2} \\
		\Rightarrow N^*(I) &= \frac{N^*(0)}{1+\frac{I}{I_s}}
	\end{align*}
	where
	\begin{align*}
		N^*(0) &= R_2\tau_2 - \frac{g_2}{g_1}R_1\tau_1 - \frac{g_2}{g_1}A_{21}\tau_1 R_2\tau_2 \\
		&= R_2\tau_2 \sbracket{1-\frac{g_2}{g_1}A_{21}\tau_1} - \frac{g_2}{g_1}R_1\tau_1 \\[1em]
		I_s &= \sbracket{\frac{\sigma_{21}}{\hbar\omega} \underbracket{\tau_2 - \frac{g_2}{g_1}\tau_1 + \frac{g_2}{g_1}A_{21}\tau_1\tau_2}_{\tau_R}}^{-1} \\
		&= \frac{\hbar\omega}{\sigma_{21}\tau_R}
	\end{align*}
	
	So gain coefficient:
	\begin{align*}
		\alpha(I) &= N^*(I)\sigma_{21} \\
		&= \frac{\alpha(0)}{1+\frac{I}{I_s}}
	\end{align*}
	with $\alpha(0)=N^*(0)\sigma_{21}$.
	
	So $\diagderi{I}{z}=\alpha(I)I$ from before, but we integrated over all $\omega$ to get $I$.
	
	\newpage
	\part Sketch of the spherical laser: \image{.3\linewidth}{q3-spherical-laser}
	Uniform population density $\Rightarrow$ $N_i(r)=N_i(R)\rbracket{\dfrac{r}{R}}^3$
	
	Spherical symmetry means that we can reuse $\diagderi{I}{z}$ from before:
	\begin{equation*}
		\deri{I}{r} = \underbracket{N^*(I,r)\sigma_{21}}_{\alpha(I)\rbracket{\frac{r}{R}}^3} I
	\end{equation*}
	with $N^*(I,r) = N^*(I) \rbracket{\dfrac{r}{R}}^3$.
	
	\part So now we have:
	\begin{align*}
		\frac{1}{I} + \frac{1}{I_s} \inftsml{I} &= \alpha(0) \rbracket{\frac{r}{R}}^3 \inftsml{r} \\
		\ln\rbracket{\frac{I}{I_0}} + \frac{I-I_0}{I_s} &= \frac{\alpha(0)}{4R^3} r^4
	\end{align*}
	
	Since $I(R)=0.1I_s \ll I_s$, we may ignore the linear term:
	\begin{equation*}
		I = I_0 e^{\frac{\alpha(0)}{4R^3}r^4}
	\end{equation*}
	Hence power $\mathcal{P}(r)=I(r)\cdot 4\pi r^2$ increases exponentially in the sphere.
	
	\part Differentiating:
	\begin{equation*}
		\deri{I}{r} = \frac{4I_0\alpha(0)}{4R^3} r^3 e^{\frac{\alpha(0)}{4R^3}r^4}
	\end{equation*}
	
	It is clear that when $r=0$, $\diagderi{I}{r}=0$ and it constitutes a minimum since $\exp$ is a monotonically increasing function.
	
	\part When $r=0$, $I=I_0$ which should be non-zero for lasing to occur.
	This also implies that for a spherical laser medium, a spontaneous emission is required to initiate population inversion -- which makes it difficult to achieve under normal conditions.
	
	However, astronomical lasers tend to have extreme conditions, e.g. high pressure or temperature that make such lasing possible.
\end{parts}