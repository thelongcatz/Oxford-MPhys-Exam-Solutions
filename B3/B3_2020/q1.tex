\draft
\begin{parts}
	\part By considering a quantity called central field $S(r)$, the Hamiltonian may be written as
	\begin{equation*}
		\hat{H}=\hat{H}_\textnormal{CF}+\Delta\hat{H}_\textnormal{RE}
	\end{equation*}
	where $\hat{H}_\textnormal{CF}=\sum_i \sbracket{\frac{\hat{\mathbf{p}}_i^2}{2m} - \frac{Ze^2}{4\pi\permittivity\hat{r}_i} + S(r_i)}$ is central field, $\Delta\hat{H}_\textnormal{RE}=\sum_i \sbracket{-S(r_i) + \sum_{j>i}\frac{e^2}{4\pi\permittivity\hat{r}_{ij}}}$ is residual electrostatic.
	
	This approximation is called central field approximation where a suitable choice of $S(r)$ minimises $\Delta\hat{H}_\textnormal{RE}$.
	The Hamiltonian is then separable into different equations for $\electron$, giving electronic configuration with quantum numbers:
	$n$ the principal quantum number, and $l$ the orbital angular momentum.
	
	For $\Delta\hat{H}_\textnormal{RE}$, it couples $\mathbf{l}$'s of different $\electron$, however since it is an internal interaction, the total orbital angular momentum $\mathbf{L}$ is conserved, making $L$ a good quantum number.
	Likewise, sinee $\Delta\hat{H}_\textnormal{RE}$ does not act on total $\electron$ spin $\mathbf{S}$, $S$ is also a good quantum number.
	The labelling of eigenstates as $\ket{LM_L SM_S}$ or $\ket{LSJM_J}$ is called $LS$ coupling.
	
	\part At small distances, the potential is simply $-\dfrac{Ze^2}{4\pi\permittivity r}$ as there is no screening effect.
	
	At large distances, however, the potential is $-\dfrac{e^2}{4\pi\permittivity r}$ as the $\electron$'s screen the nuclear charge, making an effective charge of $+e$.
	\image{.8\linewidth}{q1-screening-potential}
	
	\newpage
	\part $\electron$ density $n(p) = \dfrac{8\pi}{3h^3}p^3$ %$\Longrightarrow$ Charge density $\rho=-en$
	
%	$\electron$ energy: $\dfrac{p^2}{2m} = \dfrac{Ze^2}{4\pi\permittivity r}$
%	\begin{equation*}
%		\Rightarrow \rho = -e\cdot\frac{8\pi}{3h^3}\cdot\rbracket{\frac{2mZe^2}{4\pi\permittivity r}}^{3/2}
%	\end{equation*}
%	
%	Maxwell's equations give:
%	\begin{align*}
%		\bm{\nabla}\cdot\mathbf{E} &= \frac{\rho}{\permittivity} \\
%		\Rightarrow -\nabla^2 V &= \frac{\rho}{\permittivity} \\
%		\frac{1}{r^2} \pdiff{r}\rbracket{r^2 \pderi{V}{r}} &= \frac{8\pi e^2}{3\permittivity h^3} \rbracket{\frac{2mZ}{4\pi\permittivity r}}^{3/2} \\
%		r^2 \pderi{V}{r} &= \frac{2}{3} r^{3/2}  \frac{8\pi e^2}{3\permittivity h^3} \rbracket{\frac{2mZ}{4\pi\permittivity}}^{3/2} \\
%		V &= \frac{4}{3} r^{1/2} \frac{8\pi e^2}{3\permittivity h^3} \rbracket{\frac{Zm}{2\pi\permittivity}}^{3/2} \\
%		&= \frac{eZ}{4\pi\permittivity r} \cdot \underbracket{\frac{4}{3} r^{3/2} \frac{8\pi e^2}{3\permittivity h^3} \rbracket{\frac{Z^{1/2}m^{3/2}}{\sqrt{2}\pi^{3/2}\permittivity^{3/2}}}}_{\chi(r)}
%	\end{align*}
%	
%	Next we have:
%	\begin{align*}
%		\deri[2]{\chi}{r} &= \diff{r}\sbracket{2r^{1/2} \frac{8\pi e^2}{3\permittivity h^3}\cdot\frac{Z^{1/2}m^{3/2}}{\sqrt{2}\pi^{3/2}\permittivity^{3/2}}} \\
%		\frac{\chi^{3/2}}{x^{1/2}} &= r^{-1/2} \frac{8\pi e^2}{3\permittivity h^3}\cdot\frac{Z^{1/2}m^{3/2}}{\sqrt{2}\pi^{3/2}\permittivity^{3/2}} \\
%		\xRightarrow{\div\chi} \frac{\chi^{1/2}}{x^{1/2}} &= \frac{3}{4} r^{-2} \\
%		\Rightarrow x &= \frac{r}{b} = \frac{16}{9} r^2 \chi \\
%		\Rightarrow b &= \frac{9}{16r\chi} \\
%		&= \frac{9}{16} \cdot \frac{3}{4} r^{-5/2} \cdot \frac{3\permittivity h^3}{8\pi e^2} \rbracket{\frac{\sqrt{2}\pi^{3/2}\permittivity^{3/2}}{Z^{1/2}m^{3/2}}}
%	\end{align*}
	
%	Or try:
	\begin{align*}
		\frac{p^2}{2m} &= eV \\
		\Rightarrow \rho &= -\frac{8\pi}{3h^3} \rbracket{2meV}^{3/2}
	\end{align*}
	
	Maxwell: $\nabla^2 V = \dfrac{8\pi}{3\permittivity h^3}\rbracket{2meV}^{3/2}$
	
	Ansatz $V(r) = \dfrac{\chi(r)Ze}{4\pi\permittivity r}$:
	\begin{align*}
		\Rightarrow \frac{1}{r^2} \pdiff{r}\rbracket{r^2 \pderi{V}{r}} &= \frac{1}{r^2} \pdiff{r}\rbracket{\dot{\chi}\frac{Zer}{4\pi\permittivity} - \chi\frac{Ze}{4\pi\permittivity}} \\
		&= \ddot{\chi}\frac{Ze}{4\pi\permittivity r} + \cancel{\dot{\chi}\frac{Ze}{4\pi\permittivity r^2}} - \cancel{\dot{\chi}\frac{Ze}{4\pi\permittivity r^2}} \\
		&= \ddot{\chi}\frac{Ze}{4\pi\permittivity r}
	\end{align*}
	
	RHS reads:
	\begin{equation*}
		\frac{8\pi e}{3\permittivity h^3} \rbracket{2me \frac{\chi Ze}{4\pi\permittivity r}}^{3/2} = \chi^{3/2} \cdot \frac{16\sqrt{2}\pi e^{5/2} Z^{3/2}}{23\permittivity^{5/2}\pi^{3/2}h^3 r^{3/2}}
	\end{equation*}
	
	So:
	\begin{align*}
		\Rightarrow \ddot{\chi} &= \chi^{3/2} \frac{2\sqrt{2}e^{5/2}Z^{1/2}}{\frac{3}{4}\pi^{-1/2}\permittivity^{3/2}h^3 r^{1/2}} \\
		&= \chi^{3/2} \frac{8\sqrt{2}\pi^{1/2}e^{5/2}Z^{1/2}}{3\permittivity^{3/2}h^3 r^{1/2}} \\
		&= \chi^{3/2} \rbracket{\frac{b}{r}}^{1/2}
	\end{align*}
	where $b=\dfrac{128\pi e^5 Z}{9\permittivity^3 h^6}$.
	
	\part We have assumed that the $\electron$ behaves like a Fermi gas, but the shell structure means that the distribution of $\electron$ is not uniform, so this model would only work for massive atoms where the distribution is approximately uniform.
	ERRATA: assumed uniform $\rho$ so need large number of $\electron$.
	
	\part Lower angular momentum states are favoured more as they reside deeper in the energy level for multiple reasons:
	lower angular momentum tends to penetrate the core $\electron$ more often, thereby experiencing lower screening and is more tightly bound to nucleus.
	Furthermore, the existence of angular momentum barrier as $\dfrac{L^2}{2mr^2}$ means that lower angular momentum would contribute to having lower energy for similar penetration.
	
	ADDENDUM:
	
	$4s$ before $3d$: minimise ang mtm to minimise centrifugal barrier, lower $l$ lower potential.
\end{parts}
