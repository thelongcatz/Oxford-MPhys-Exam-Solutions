\draft
\begin{parts}
	\part $A\mathbf{I}\cdot\mathbf{J}$: the hyperfine interaction where the intrinsic magnetic moments of nucleus and $\electron$ interact via magnetic dipole interaction.
	
	$g_J \bohrmagneton \mathbf{J}\cdot\mathbf{B}$: external magnetic interaction between magnetic moments of $\electron$ and the external magnetic field.
	$\mathbf{J}$ is used in place of $\sum_i \mathbf{l}_i = \mathbf{L}$ and $\sum_i \mathbf{s}_i = \mathbf{S}$ by Wigher-Eckart theorem.
	
	$g_I \bohrmagneton \mathbf{I}\cdot\mathbf{B}$: similar to above but it is the nuclear magnetic moments which interact instead.
	
	\part In the weak field limit, $A\mathbf{I}\cdot\mathbf{J} \gg g_I \bohrmagneton \mathbf{I}\cdot\mathbf{B}$ (nuclear term negligible).
	
	Hyperfine interaction has typical energies in microwave range, e.g. $\lambda = \SI{21}{\centi\metre}$.
	
	So:
	\begin{align*}
		\bohrmagneton B \sim \frac{hc}{\lambda} \\
		\Rightarrow B \sim \frac{hc}{\lambda\bohrmagneton} = \SI{0.10}{\tesla}
	\end{align*}
	within this the weak field limit applies.
	
	\part Ignoring nuclear term since it is negligible ($\nuclearmagneton=\dfrac{m_e}{m_\textnormal{p}} \ll \bohrmagneton$):
	\begin{equation*}
		\Delta\hat{H} = A\mathbf{I}\cdot\mathbf{J} + g_J \bohrmagneton \mathbf{J}\cdot\mathbf{B}
	\end{equation*}
	
	Under weak field, the dominant interaction is the hyperfine interaction, which means that $\mathbf{F}=\mathbf{I}+\mathbf{J}$ is conserved and so $F$ is a good quantum number.
	
	By Wigner-Eckart, $\mathbf{J}\to\dfrac{\mathbf{J}\cdot\mathbf{F}}{F^2}\mathbf{F}$ so:
	\begin{align*}
		\Delta\hat{H} &= A \cdot \frac{1}{2}\sbracket{\hat{\mathbf{F}}^2 - \hat{\mathbf{I}}^2 - \hat{\mathbf{J}}^2} + g_J \bohrmagneton \frac{\hat{J}^2+\hat{\mathbf{I}}\cdot\hat{\mathbf{J}}}{F^2}\mathbf{F}\cdot\mathbf{B} \\
		\Rightarrow \Delta E &= \frac{A}{2} \underbracket{\sbracket{F(F+1)-I(I+1)-J(J+1)}}_{K} \\
		&\qquad + \underbracket{g_J \frac{J(J+1)+\frac{1}{2}\sbracket{F(F+1)-I(I+1)-J(J+1)}}{F(F+1)}}_{g_F} \bohrmagneton M_F B
	\end{align*}
	
	\part $J=3/2$, $I=3$.
	So the possible values of $F$ are:
	\begin{align*}
		\frac{3}{2} &\longrightarrow K=-12 \qquad\qquad \frac{g_F}{g_J}=-\frac{3}{5} \\
		\frac{5}{2} &\longrightarrow K=-7 \qquad\qquad \frac{g_F}{g_J}=\frac{1}{35} \\
		\frac{7}{2} &\longrightarrow K=0 \qquad\qquad \frac{g_F}{g_J}=\frac{5}{21} \\
		\frac{9}{2} &\longrightarrow K=9 \qquad\qquad \frac{g_F}{g_J}=\frac{1}{3}
	\end{align*}
	\image{.8\linewidth}{q2-mf-degen-lifting}
	
	\part In the strong field case, the dominant interaction is the external Hamiltonian, making $M_I$ and $M_J$ good quantum numbers instead.
	\begin{equation*}
		\Rightarrow\Delta E = AM_IM_J + g_J\bohrmagneton M_JB - g_I\nuclearmagneton M_IB
	\end{equation*}
	So for hydrogen ground state $1s\, {}^2 S_{1/2}$:
	\begin{align*}
		\Delta E\rbracket{J_z=\frac{1}{2}, I_z=-\frac{1}{2}} = -\frac{3}{4}A + \frac{B}{2} \sbracket{g_J\bohrmagneton + g_I\nuclearmagneton} \\
		\Delta E\rbracket{J_z=-\frac{1}{2}, I_z=-\frac{1}{2}} = \frac{3}{4}A + \frac{B}{2} \sbracket{-g_J\bohrmagneton + g_I\nuclearmagneton} \\
		\Delta E\rbracket{J_z=\frac{1}{2}, I_z=\frac{1}{2}} = \frac{3}{4}A + \frac{B}{2} \sbracket{g_J\bohrmagneton - g_I\nuclearmagneton}
	\end{align*}
	
	So:
	\begin{align*}
		\frac{\Delta E\rbracket{J_z=\frac{1}{2}, I_z=-\frac{1}{2}} + \Delta E\rbracket{J_z=-\frac{1}{2}, I_z=-\frac{1}{2}}}{\Delta E\rbracket{J_z=\frac{1}{2}, I_z=-\frac{1}{2}} + \Delta E\rbracket{J_z=\frac{1}{2}, I_z=\frac{1}{2}}} &= \frac{\cfrac{B}{2}\sbracket{2g_I\nuclearmagneton}}{\cfrac{B}{2}\sbracket{2g_J\bohrmagneton}} \\
		&= \frac{g_p \nuclearmagneton}{g_s \bohrmagneton}
	\end{align*}
	for hydrogen (\element{H}{1}) specifically.
\end{parts}
