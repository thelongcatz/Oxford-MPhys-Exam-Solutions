\draft
\begin{parts}
	\part As X-ray transition involves core $\electron$, which have symmetric wavefunctions, the energy transition follows the Rydberg model where $E = -hcR_\infty Z^2/n^2$ it, however due to shielding, a parameter $\sigma$ is added to represent its effect on reducing the effective potential.
	
	Thus
	\begin{equation*}
		E_{nm} = hcR_\infty \sbracket{\frac{(Z-\sigma_n)^2}{n^2} - \frac{(Z-\sigma_m)^2}{m^2}}
	\end{equation*}
	with $\sigma$ being different for each $n$.
	
	\part Conservation of energy gives $E_\textnormal{x-ray} = E_{\electron} + T_{\electron}$ where $E_{\electron}$ is the energy cost of ejecting the $\electron$, $T_{\electron}$ is the kinetic energy of the $\electron$.
	
	\begin{align*}
		E_{\electron} &= E_\textnormal{x-ray} - T_{\electron} \\
		&= \SI{870}{\electronvolt}, \SI{48.5}{\electronvolt}, \SI{21.6}{\electronvolt}
	\end{align*}
	
	Further emission is due to the Auger effect, which causes electrons to be knocked off from the valence shell instead of photons.
	
	Double ionisation is due to the Auger effect kicking more $\electron$.
	
	For $L$ shell $\electron$,
	\begin{equation*}
		E = -hcR_\infty \frac{10^2}{2^2} = \SI{-340.15}{\electronvolt}
	\end{equation*}
	So the minimum energy required for X-ray generation is $\SI{340.15}{\electronvolt}$.
\end{parts}