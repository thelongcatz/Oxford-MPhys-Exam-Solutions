Density of states of a \underline{non-parabolic} dispersion relation.

Despite the seeming difficulty, it is perfectly doable with the hints given\footnote{Though sadly the author rolled a 1 in perception and failed to spot the boldface $\mathbf{k}$ in the hint given and lost 7 easy marks in part d\dots}.

\begin{parts}
	\part Straightforward calculation from the fact that we have an \textit{isotropic} band, thus
	\begin{gather*}
		A(k_{||}) = \pi k^{2} \\
		\Rightarrow \pderi{A}{k} = 2\pi k
	\end{gather*}
	
	By chain rule, we have $\pdderi{A}{\epsilon} = \pdderi{A}{k} \dderi{k}{\epsilon}$:
	\begin{align}
		\omega_c &= \frac{2\pi eB}{\hbar^2} \times \deri{\epsilon}{k} \times \frac{1}{2\pi k} \notag \\
		&= \frac{eB}{\hbar^2} \frac{\mathrm{d}\epsilon / \mathrm{d}k}{k} \label{eqn:q2-omega-c}
	\end{align}
	
	\part Implicitly differentiating the given dispersion gives (here $\epsilon^\prime = \mathrm{d}\epsilon / \mathrm{d}k$):
	\begin{gather}
		\epsilon^\prime + \frac{2\epsilon\epsilon^\prime}{E_g} = \frac{\hbar^2 k}{m_0^*} \notag \\
		\Rightarrow \epsilon^\prime = \frac{\hbar^2 k}{m_0^*} \sbracket{1 + \frac{2\epsilon}{E_g}}^{-1} \label{eqn:q2-de-dk}
	\end{gather}
	
	Plugging \eqref{eqn:q2-de-dk} into \eqref{eqn:q2-omega-c} then gives:
	\begin{align*}
		\omega_c &= \frac{eB}{\hbar^2 k} \, \frac{\hbar^2 k}{m_0^*} \sbracket{1 + \frac{2\epsilon}{E_g}}^{-1} \\
		&= \frac{eB}{m_0^*} \, \frac{1}{1+2\epsilon / E_g}
	\end{align*}
	
	\part Here we need to use the definition of cyclotron frequency $\omega_c = \dfrac{eB}{m_c}$, where $m_c$ is the cyclotron mass.
	Comparing the two expression then gives:
	\begin{equation*}
		m_c = m_0^* \sbracket{ 1+\frac{2\epsilon}{E_g} }
	\end{equation*}
	
	\part Following the hints and taking care that we have a \textbf{3D} (see footnote \frownie) integral:
	\begin{align*}
		g(E) &= \frac{2}{V} \, \sum_{\mathbf{k}} \delta\sbracket{E - \epsilon(\mathbf{k})} \\
		&\rightarrow \frac{2}{V} \times \frac{V}{\rbracket{2\pi}^3} \int \inftsml{{}^{3}\mathbf{k}} \  \delta\sbracket{E - \epsilon(\mathbf{k})} \textnormal{\hspace{1em}by continuum approximation} \\
		&= \frac{2}{\rbracket{2\pi}^3} \int \inftsml{k} \  4\pi k^2 \  \delta\sbracket{E - \epsilon(\mathbf{k})}
	\end{align*}
	
	From \eqref{eqn:q2-de-dk} we have $k \inftsml{k} = \dfrac{m_0^*}{\hbar^2}\sbracket{ 1 + \dfrac{2\epsilon}{E_g}} \inftsml{\epsilon}$, combining this together with $k$ from the dispersion then gives:
	\begin{align*}
		g(E) &= \frac{1}{\pi^2} \int \inftsml{\epsilon} \,
		\frac{\sqrt{2} m_0^{*^{3/2}}}{\hbar^3}
		\rbracket{1 + \frac{2\epsilon}{E_g}}
		\sqrt{\epsilon\sbracket{1 + \frac{\epsilon}{E_g}}}
		\delta\sbracket{E - \epsilon} \\
		&= \underbracket{\frac{\sqrt{2} m_0^{*^{3/2}}}{\pi^2 \hbar^3}}_{\alpha} \rbracket{1 + \frac{2E}{E_g}} \sqrt{E\sbracket{1 + \frac{E}{E_g}}}
	\end{align*}
	as per the properties of the delta function.
	
	\part In the limit of $E / E_g \ll 1$, we may drop the $\epsilon / E_g$ term in the dispersion to get the parabolic dispersion immediately.
	Similar result follows for the cyclotron mass.
	
	For d.o.s., we may drop higher order terms in the following expansion:
	\begin{align*}
		\lim\limits_{E/E_g \rightarrow 0} g(E) &= \lim\limits_{E/E_g \rightarrow 0} \alpha \sqrt{E} \rbracket{1+\frac{2E}{E_g}} \sbracket{1+\frac{E}{2E_g}+\ldots} \\
		&= \lim\limits_{E/E_g \rightarrow 0} \alpha \sqrt{E} \rbracket{1 + \frac{5}{2} \, \frac{E}{E_g} + \ldots} \\
		&= \alpha \sqrt{E}
	\end{align*}
	And we have now recovered the parabolic d.o.s.!
	
	\part For the parabolic approximation to fit within the dispersion above with an accuracy of $10\unit{\percent}$, we construct the following ratio (for that $g(E)$ determines almost all properties of a material):
	\begin{gather}
		\frac{\alpha \rbracket{1 + \frac{2E}{E_g}} \sqrt{E\sbracket{1 + \frac{E}{E_g}}}}{\alpha \sqrt{E}} = 110\unit{\percent} \notag \\
		\Rightarrow \rbracket{1 + \frac{2E}{E_g}} \sqrt{1 + \frac{E}{E_g}} = 1.1 \label{eqn:q2-ratio}
	\end{gather}
	
	From here we could meticulously solve the cubic equation, however we can binomial expand \eqref{eqn:q2-ratio} in hopes of the results being small\footnote{As reassurance, the actual roots are pretty small: $0.03946$ and the unphysical roots of $-1.01973 \pm 0.53910i$.}:
	\begin{gather*}
		\rbracket{1+\frac{2E}{E_g}} \sbracket{1+\frac{E}{2E_g}} = 1.1 \\
		\rbracket{\frac{E}{E_g}}^2 + \frac{5}{2} \, \frac{E}{E_g} - 0.1 = 0 \\
		\frac{E}{E_g} = \frac{-\diagfrac{5}{2} \pm \sqrt{\diagfrac{25}{4} + 0.4}}{2} \\
		= 0.0394
	\end{gather*}
	where the unphysical negative root is dropped.
	
	For the final part of this question, we simply plug in the value of the band gap for each semiconductor.
	
	InAs:
	\begin{align*}
		E &= 0.0394 \times \SI{0.35}{\electronvolt} \\
		&= \SI{0.0138}{\electronvolt}
	\end{align*}
	
	GaSb:
	\begin{align*}
		E &= 0.0394 \times \SI{0.73}{\electronvolt} \\
		&= \SI{0.0287}{\electronvolt}
	\end{align*}
	
	GaAs:
	\begin{align*}
		E &= 0.0394 \times \SI{1.42}{\electronvolt} \\
		&= \SI{0.0559}{\electronvolt}
	\end{align*}
\end{parts}