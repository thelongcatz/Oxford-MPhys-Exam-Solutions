Classic Jaynes-Cummings question. The recent topic of separable/non-separable vector spaces from 2023 makes this question a rather familiar one\footnote{That being said, the author still had the ingenuity of quoting double the value of Rabi frequency and muddling up the last part!}.
\begin{parts}
	\part We begin by considering a cavity with (single mode) EM field energy density:
	\begin{equation*}
		U = \frac{1}{2}\epsilon_0 E^2 + \frac{1}{2\mu_0} B^2
	\end{equation*}
	
	By second quantisation, we may make the (up to some constant that makes up $\hbar\omega$) substitutions $E \rightarrow \diagfrac{1}{\sqrt{2}}(a+a^\dag)$ and $B \rightarrow \diagfrac{i}{\sqrt{2}}(a-a^\dag)$.
	The resulting Hamiltonian is then:
	\begin{align*}
		\mathcal{H} &= \frac{1}{4} \hbar\omega \sbracket{\rbracket{a + a^\dag}^2 - \rbracket{a - a^\dag}^2} \\
		&= \frac{\hbar\omega}{4} \sbracket{2 \underbracket{a a^\dag}_{1+a^\dag a} + 2a^\dag a} \\
		&= \hbar\omega \rbracket{a^\dag a + \frac{1}{2}}
	\end{align*}
	This corresponds nicely to a quantum harmonic oscillator -- in fact the offset of $\diagfrac{1}{2}$ suggests that energy is non-zero even when there is no photon!
	
	\part Usual Jaynes-Cummings Hamiltonian \textit{on resonance}, be careful about the Rabi frequency though\footnote{Tip: since Rabi flopping is an observable quantity, the frequency that goes into the amplitude should be halved so that the correct frequency is recovered upon squaring (you can see that $\cos^2$ is frequency doubling by writing it in exponentials).}:
	\begin{equation}
		\mathcal{H} = \underbracket{\hbar\omega \frac{\sigma_z}{2}}_{\textnormal{Atomic energy}} + \underbracket{\hbar\omega a^\dag a}_{\textnormal{Photonic energy}} + \underbracket{\frac{\hbar\Omega_a}{2}(a^\dag \sigma_- + a \sigma_+)}_{\textnormal{Mode $a$ coupling}} + \underbracket{\frac{\hbar\Omega_b}{2}(a^\dag \sigma_- + a \sigma_+)}_{\textnormal{Mode $b$ coupling}}
	\end{equation}
	
	\part For this part we set $\Omega_b = 0$, then solve the TDSE with Jaynes-Cummings Hamiltonian.
	
	My personal favourite way of solving this is by diagonalising the Hamiltonian (here I ignored mode $b$ for clarity):
	\begin{gather*}
		H = \bordermatrix{
			& \bra{\textnormal{e},\, 0_a} & \bra{\textnormal{g},\, 1_a} \cr
			& \dfrac{\hbar\omega}{2} & \dfrac{\hbar\Omega_a}{2} \cr
			& \dfrac{\hbar\Omega_a}{2} & \hbar\omega - \dfrac{\hbar\omega}{2}
		} \\
		\Rightarrow \begin{vmatrix}
			\dfrac{\hbar\omega}{2} - E & \dfrac{\hbar\Omega_a}{2} \\[1em]
			\dfrac{\hbar\Omega_a}{2} & \dfrac{\hbar\omega}{2} - E
		\end{vmatrix} = 0 \\
		\Rightarrow E_\pm = \frac{\hbar\omega}{2} \pm \frac{\hbar\Omega_a}{2}
	\end{gather*}
	And the corresponding dressed states are $\ket{\pm} = \diagfrac{1}{\sqrt{2}} \bigl(\ket{\textnormal{e},\, 0_a} \pm \ket{\textnormal{g},\, 1_a}\bigr)$.
	
	TDSE then tells us the general solution would be:
	\begin{equation*}
		\ket{\psi(t)} = A\exp\rbracket{-\frac{iE_+ t}{\hbar}} \ket{+} + B\exp\rbracket{-\frac{iE_- t}{\hbar}} \ket{-}
	\end{equation*}
	Substituting the initial condition of $\ket{\psi(0)} = \ket{\textnormal{e},\, 0_a} = \diagfrac{1}{\sqrt{2}} \bigl(\ket{+} + \ket{-}\bigr)$ then gives:
	\begin{equation*}
		A = B = \frac{1}{\sqrt{2}}
	\end{equation*}
	Further simplification then gives the familiar Rabi flopping:
	\begin{align}
		\ket{\psi(t)} &=
		\begin{aligned}[t]
			\frac{1}{\sqrt{2}} \exp\rbracket{-\frac{i\omega t}{2}} &\sbracket{\frac{1}{\sqrt{2}} \rbracket{\exp\rbracket{-\frac{\Omega_a t}{2}} + \exp\rbracket{\frac{\Omega_a t}{2}}} \ket{\textnormal{e},\, 0_a} \right. \\
			&\left. + \frac{1}{\sqrt{2}} \rbracket{\exp\rbracket{-\frac{\Omega_a t}{2}} - \exp\rbracket{\frac{\Omega_a t}{2}}} \ket{\textnormal{g},\, 1_a}}
		\end{aligned} \notag \\
		&= \exp\rbracket{-\frac{i\omega t}{2}} \sbracket{\cos\rbracket{\frac{\Omega_a t}{2}} \ket{\textnormal{e},\, 0_a} - i\sin\rbracket{\frac{\Omega_a t}{2}} \ket{\textnormal{g},\, 1_a}}
		\label{eqn:q5-psi-t}
	\end{align}
	
	\part At $t = \tau = \pi / 2\Omega_a$, \eqref{eqn:q5-psi-t} becomes:
	\begin{equation*}
		\ket{\psi(\tau)} = \exp\rbracket{-\frac{i\omega\pi}{4\Omega_a}} \sbracket{\cos\rbracket{\frac{\pi}{4}} \ket{\textnormal{e},\, 0_a} -i\sin\rbracket{\frac{\pi}{4}} \ket{\textnormal{g},\, 1_a}}
	\end{equation*}
	
	Now we turn mode $a$ off and mode $b$ on, now we have coupling between the states $\ket{\textnormal{e},\, 0_b}$ and $\ket{\textnormal{g},\, 1_b}$.
	
	Since $a$ and $b$ do not have any direct coupling, the state $\ket{\textnormal{g},\, 1_a}$ remains stationary.
	
	Repeating the same procedure as the previous part then yields:
	\begin{equation*}
		\ket{\psi^\prime (t)} = C\exp\rbracket{-\frac{iE_+^\prime t}{\hbar}} \ket{+^\prime} + D\exp\rbracket{-\frac{iE_-^\prime t}{\hbar}} \ket{-^\prime}
	\end{equation*}
	where the primed variables are of mode $b$ instead of $a$.
	
	However this time we have a different initial state of $\ket{\psi^\prime (0)} = \diagfrac{1}{\sqrt{2}} \ket{\textnormal{e},\, 0_b} = \diagfrac{1}{2} \bigl(\ket{+^\prime} + \ket{-^\prime}\bigr)$:
	\begin{equation*}
		C = D = \frac{1}{2}
	\end{equation*}
	
	So the new state at time $t$ shall be:
	\begin{align*}
		\ket{\Psi (t)} = \frac{1}{\sqrt{2}} \exp\rbracket{-\frac{i\omega t}{2}} &\sbracket{\cos\rbracket{\frac{\Omega_b t}{2}} \ket{\textnormal{e},\, 0_a,\, 0_b} - i\sin\rbracket{\frac{\Omega_b t}{2}} \ket{\textnormal{g},\, 0_a,\, 1_b} \right. \\
		&\left. -i \exp\rbracket{-\frac{i\omega\pi}{4\Omega_a}} \ket{\textnormal{g},\, 1_a,\, 0_b}}
	\end{align*}
	
	Given the atom is measured to be in the ground state, the state of the field is then the following mixed state:
	\begin{equation*}
		\rho = \frac{\diagfrac{1}{2}}{\diagfrac{1}{2} [\sin^2(\Omega_b t/2) + 1]} \sbracket{\sin^2\rbracket{\frac{\Omega_b t}{2}} \ket{0_a,\, 1_b} \bra{0_a,\, 1_b} + \ket{1_a,\, 0_b} \bra{1_a,\, 0_b}}
	\end{equation*}
	where the denominator simply normalises the probabilities of the density matrix.
	
	To examine if they are separable, we shall inspect the density matrices of the subsystems:
	\begin{align*}
		\rho_{a/b} &= \textnormal{tr}_{a/b}\,\rho \\
		&= \frac{1}{\sin^2(\Omega_b t/2) + 1} \sbracket{\sin^2\rbracket{\frac{\Omega_b t}{2}} \ket{0_a/1_b} \bra{0_a/1_b} + \ket{1_a/0_b} \bra{1_a/0_b}}
	\end{align*}
	
	Note that by symmetry, $\rho_a$ and $\rho_b$ share the same purity:
	\begin{equation*}
		\textnormal{tr}(\rho_{a/b}^2) = \frac{\sin^4 (\Omega_b t/2) + 1}{[\sin^2(\Omega_b t/2) + 1]^2}
	\end{equation*}
	
	The states are separable when the purity is 1:
	\begin{gather*}
		\sin^4 \rbracket{\frac{\Omega_b t}{2}} + 2\sin^2 \rbracket{\frac{\Omega_b t}{2}} + 1 = \sin^4 \rbracket{\frac{\Omega_b t}{2}} + 1 \\
		\Rightarrow \sin\rbracket{\frac{\Omega_b t}{2}} = 0 \\
		\Rightarrow t = \frac{2n\pi}{\Omega_b}
	\end{gather*}
	where $n\in\mathbb{Z}$. At other times the states are entangled.
\end{parts}