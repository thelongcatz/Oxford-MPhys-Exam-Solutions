\textbf{(DRAFT)}
Rabi flopping in the context of atomic/ionic quantum computing.
\begin{parts}
	\part Mathematical proof as to why the evolution of a closed quantum system is unitary.
	
	From the spectral theorem, we may immediately quote that for an operator $\mathcal{H}$ that commutes with its conjugate, there exists a complete set\footnote{Mathematically referred to as the \textit{spectrum} of the operator. Surprise!} of orthonormal eigenbasis:
	\begin{equation}
		\mathcal{H} = \sum_{i} \lambda_i \ket{i}\bra{i}
		\label{eqn:q3-spectral}
	\end{equation}
	
	We then have the TISE which relates \eqref{eqn:q3-spectral} to the following:
	\begin{equation*}
		\mathcal{H}\ket{\psi} = E\ket{\psi}
		\label{eqn:q3-tise}
	\end{equation*}
	where we rewrite $\ket{i}\rightarrow\ket{\psi}$, and $E$ takes the role of energy of state $\ket{\psi}$.
	
	We require energy to be real to make physical sense, hence the eigenvalues of $\mathcal{H}$ must be real.
	
	This means that $\mathcal{H}$ is Hermitian, i.e. $\mathcal{H} = \mathcal{H}^\dag$.
	Hence the propagator $U = \exp(-i\mathcal{H}t/\hbar)$ satisfies the following relation:
	\begin{align*}
		UU^\dag &= \exp\left(-\frac{i \mathcal{H} t}{\hbar}\right) \exp\left(\frac{i \mathcal{H}^\dag t}{\hbar}\right) \\
		&= \exp\left(-\frac{i \mathcal{H} t}{\hbar} + \frac{i \mathcal{H} t}{\hbar}\right) \textnormal{\hspace{1em}since }[\mathcal{H},\, \mathcal{H}^\dag] = 0 \\
		&= \exp(0) \\
		&= \mathds{1}
	\end{align*}
	Thus $U$ is unitary.
	
	\part Diagonalising $\mathcal{H}_2$ gives:
	\begin{gather}
		\begin{vmatrix}
			-\lambda & \hbar V/2 \\
			\hbar V/2 & \hbar\delta - \lambda
		\end{vmatrix} = 0 \notag \\
		\lambda^2 - \hbar\delta\lambda - \frac{\hbar^2 V^2}{4} = 0 \notag \\
		\lambda = \frac{\hbar\delta \pm \sqrt{\hbar^2 \delta^2 + \hbar^2 V^2}}{2}
		\label{eqn:q3-h2-eigenvalues}
	\end{gather}
	
	For the case where $V = 0$, we have:
	\begin{align*}
		\lambda = \hbar\delta \Rightarrow \textnormal{Eigenvector: }
		\begin{pmatrix}
			0 \\ 1
		\end{pmatrix}
		&& \lambda = 0 \Rightarrow \textnormal{Eigenvector: }
		\begin{pmatrix}
			1 \\ 0
		\end{pmatrix}
	\end{align*}
	
	Hence a system with initial state $\ket{0}$ or $\ket{1}$ remains stationary over time in this case.
	
	Similarly for the case where $\delta = 0$, we have:
	\begin{align*}
		\lambda = \frac{\hbar V}{2} \Rightarrow \textnormal{Eigenvector: }
		\ket{+} = \frac{1}{\sqrt{2}}
		\begin{pmatrix}
			1 \\ 1
		\end{pmatrix} \\
		\lambda = -\frac{\hbar V}{2} \Rightarrow \textnormal{Eigenvector: }
		\ket{-} = \frac{1}{\sqrt{2}}
		\begin{pmatrix}
			1 \\ -1
		\end{pmatrix}
	\end{align*}
	
	So a system undergoes Rabi flopping over time:
	\begin{equation*}
		\ket{\psi(t)} = A \ket{+} \exp\left(-\frac{iVt}{2}\right) + B \ket{-} \exp\left(\frac{iVt}{2}\right)
	\end{equation*}
	Note that $\ket{0} = \bigl(\ket{+} \;+\; \ket{-}\bigr) / \sqrt{2}$ and $\ket{1} = \bigl(\ket{+} \;-\; \ket{-}\bigr) / \sqrt{2}$.
	
	An initial state of $\ket{0}$ then gives $A = B = 1/\sqrt{2}$, hence:
	\begin{equation*}
		\ket{\psi(t)} = \cos\left(\frac{Vt}{2}\right) \ket{0} -i\sin\left(\frac{Vt}{2}\right) \ket{1}
	\end{equation*}
	Whereas an initial state of $\ket{1}$ gives $A = -B = 1/\sqrt{2}$:
	\begin{equation*}
		\ket{\psi(t)} = -i\sin\left(\frac{Vt}{2}\right) \ket{0} + \cos\left(\frac{Vt}{2}\right) \ket{1}
	\end{equation*}
	
	In the case where $V \ll \delta$, we may apply perturbation theory onto the case with $V = 0$.
	So we rewrite $\mathcal{H} = \mathcal{H}_0 + \Delta\mathcal{H}$ where $\mathcal{H}_0$ is the Hamiltonian of the unperturbed two-level system, $\Delta\mathcal{H}$ is the control Hamiltonian with the off-diagonal terms.
	
	The first order change in the state is given as:
	\begin{gather*}
		\Delta\ket{0} = \sum_{i \neq 0} \frac{\bra{i} \Delta\mathcal{H} \ket{0}}{E_{0} - E_{i}} = -\frac{V}{2\delta} \ket{1} \sim 0 \\
		\Delta\ket{1} = \sum_{i \neq 1} \frac{\bra{i} \Delta\mathcal{H} \ket{1}}{E_{1} - E_{i}} = \frac{V}{2\delta} \ket{0} \sim 0
	\end{gather*}
	
	Therefore taking the eigenvectors from the case $V = 0$ is a valid approach.
	
	We proceed to solve the TDSE with the \textit{exact} eigenvalue\footnote{Not a huge fan of the wording but this is the most logical thing I can think of!} from \eqref{eqn:q3-h2-eigenvalues} to get detuned Rabi flopping:
	\begin{align}
		\ket{\psi(t)} &= C \ket{0} \exp\left( -\frac{i\lambda_+ t}{\hbar} \right) + D \ket{1} \exp\left( -\frac{i\lambda_- t}{\hbar} \right) \notag \\
		&= C \ket{0} \exp\left( -it\,\frac{\delta + \sqrt{\delta^2 + V^2}}{2} \right) + D \ket{1} \exp\left( -it\,\frac{\delta - \sqrt{\delta^2 + V^2}}{2} \right) \notag \\
		&= \exp(-i\delta t/2) \left[ C \ket{0} \exp\left( -it \, \sqrt{\delta^2 + V^2}/2 \right) + D \ket{1} 
		\exp\left( it \, \sqrt{\delta^2 + V^2}/2 \right) \right]
		\label{eqn:q3-detuned-rabi}
	\end{align}
	
	\part The very first assumption we made to analyse an atomic/ionic quantum system is to assume that it is a two-level system.
	
	We also changed from the lab frame into the rotating frame to get rid of the oscillating exponentials.
	
	In addition, we have also invoked the rotating wave approximation where we ignored the counter-rotating component of the incoming wave: $\cos\omega t = \diagfrac{1}{2}(\mathrm{e}^{i\omega t} + \underbracket{\mathrm{e}^{-i\omega t}}_{\substack{\textnormal{counter-rotating}\\\textnormal{component}}})$.
	
	The difference between optical and microwave transitions is the lifetime of the levels involved -- as Rabi oscillation requires a fairly strongly-driven interaction to be observed, we need high-power laser to offset the inherent short lifetime of the upper optical level.
	
	On the other hand, the microwave transitions are usually electric dipole forbidden, which implies that they are of higher order and thus less likely to occur.
	In addition, focusing a microwave beam to address an atom is difficult as the Abbe limit states that the minimum spot diameter $d \gtrsim \lambda/2$.
	
	\part In Dirac notation, we may write the Hamiltonian of the three-level system as:
	\begin{equation*}
		\mathcal{H}_3 = \frac{\hbar V}{2} \Bigl( \ket{2}\bra{0} \;+\; \ket{2}\bra{1} \Bigr) + \hbar\delta \ket{2}\bra{2} + \textnormal{complex conjugate (c.c.)}
	\end{equation*}
	
	Rewriting $\ket{\pm} = \bigl(\ket{0} \pm \ket{1}\bigr)/\sqrt{2}$ then gives:
	\begin{align*}
		\mathcal{H}_3 &= \frac{\hbar V}{2} \ket{2} \times \left( \sqrt{2} \bra{+} \right) + \hbar\delta \ket{2}\bra{2} + \textnormal{c.c.} \\
		&= \hbar \bordermatrix{
			&\bra{-} & \bra{+} & \bra{2} \cr
			&0 & 0 & 0 \cr
			&0 & 0 & V/\sqrt{2} \cr
			&0 & V/\sqrt{2} & \delta
		}
	\end{align*}
	
	Note that the off-diagonal interaction only occurs between the state $\ket{+}$ and $\ket{2}$, hence we may quote the previous part \eqref{eqn:q3-detuned-rabi} for the time evolution of the state:
	\begin{align*}
		\ket{\psi(t)} = &X \ket{-} \exp(0) \\
		&+ \exp(-i\delta t/2) \left[ Y \ket{+} \exp\left( -it \, \sqrt{\delta^2 + V^2}/2 \right) + Z \ket{2} 
		\exp\left( it \, \sqrt{\delta^2 + V^2}/2 \right) \right]
	\end{align*}
	
	For an initial state of $\ket{0}$, we have $X=Y=1/\sqrt{2}$ and $Z=0$.
	Hence the evolution will be given as:
	\begin{align*}
		\ket{\psi(t)} &= \frac{1}{\sqrt{2}} \left[ \ket{-} \;+\; \ket{+} \exp\left( -it\,\frac{\delta + \sqrt{\delta^2 + V^2}}{2} \right) \right] \\
		&= \frac{1}{2} \Bigl[ \bigl(\exp(-i\Omega t)+1\bigr)\ket{0} \;+\; \bigl(\exp(-i\Omega t)-1\bigr)\ket{1} \Bigr] \\
		&= \exp\left(-\frac{i\Omega t}{2}\right) \left[ \cos\left(\frac{\Omega t}{2}\right)\ket{0} - i\sin\left(\frac{\Omega t}{2}\right) \ket{1} \right]
	\end{align*}
	
	And this is simply a detuned Rabi flopping with effective frequency $\Omega = \diagfrac{1}{2} \left(\delta + \sqrt{\delta^2 + V^2}\right)$.
		
	\part The setup above corresponds to Raman transitions where an additional third level is used to facilitate transitions between two low-lying levels where direct transition is forbidden via the two-photon process.
	
	The result shows that even though direct transition is forbidden, we may still induce Rabi flopping between the states $\ket{0}$ and $\ket{1}$.
\end{parts}