Classic Jaynes-Cummings question.
\begin{parts}
	\part Jaynes-Cummings model quantises the electric dipole interaction as follows:
	\begin{align*}
		\mathcal{H} &= \mathbf{p}\cdot\mathbf{E} \\
		&\rightarrow \left(\sigma_+ + \sigma_-\right) \left(a+a^\dag\right)
	\end{align*}
	where $\sigma_\pm$ are the raising/lowering operators for the atomic energy levels, $a$ and $a^\dag$ are annihilation/creation operators.
	
	In the rotating frame, we may write the interacting Hamiltonian as:
	\begin{align*}
		\mathcal{H}_\textnormal{int} &= \frac{\hbar\Omega}{2}\left[
		\sigma_+ a \mathnormal{e}^{-i(\omega_0 - \omega)t}
		+ \sigma_+ a^\dag \mathnormal{e}^{-i(\omega_0 + \omega)t}
		+ \sigma_- a \mathnormal{e}^{i(\omega_0 - \omega)t}
		+ \sigma_- a^\dag \mathnormal{e}^{i(\omega_0 + \omega)t}
		\right] \\
		&\approx \frac{\hbar\Omega}{2}\left[\sigma_+ a \mathnormal{e}^{-i(\omega_0 - \omega)t} + \sigma_- a^\dag \mathnormal{e}^{i(\omega_0 - \omega)t}\right] \textnormal{\hspace{1em}by RWA}
	\end{align*}
	where $\omega_0$ is the energy difference between ground and excited states of the atom, $\omega$ is the frequency of the photon, $\Omega$ is the Rabi frequency of the system.
	
	Thus the total Hamiltonian on resonance is:
	\begin{align*}
		\mathcal{H} &= \mathcal{H}_\textnormal{atom} + \mathcal{H}_\textnormal{light} + \mathcal{H}_\textnormal{int} \\
		&= \hbar\omega\frac{\sigma_z}{2} + \hbar\omega a^\dag a + \frac{\hbar\Omega}{2} \left(\sigma_+ a^\dag + \sigma_- a\right)
	\end{align*}
	
	\part TDSE: $i\hbar\frac{\partial}{\partial t} \ket{\psi} = \mathcal{H}\ket{\psi}$.
	
	In matrix notation the Hamiltonian reads:
	\begin{equation*}
		\mathcal{H} = \bordermatrix{
			& \bra{\textnormal{e, 0}} & \bra{\textnormal{g, 1}} \cr\vspace{1ex}
			& \dfrac{\hbar\omega}{2} & \dfrac{\hbar\Omega}{2} \cr
			& \dfrac{\hbar\Omega}{2} & -\dfrac{\hbar\omega}{2} + \hbar\omega
		}
	\end{equation*}
	
	Diagonalising $\mathcal{H}$ then gives:
	\begin{gather*}
		\begin{vmatrix}
			\dfrac{\hbar\omega}{2}-E & \dfrac{\hbar\Omega}{2} \\[1em]
			\dfrac{\hbar\Omega}{2} & \dfrac{\hbar\omega}{2}-E
		\end{vmatrix} = 0 \\
		\Rightarrow E_\pm = \frac{\hbar\omega}{2} \pm \frac{\hbar\Omega}{2}
	\end{gather*}
	where $E_\pm$ are the energies of the dressed states: $\ket{\pm} = \diagfrac{1}{\sqrt{2}}\bigl[\ket{\textnormal{e, 0}}\pm\ket{\textnormal{g, 1}}\bigr]$.
	
	Thus the system evolves in the Schrödinger picture as:
	\begin{equation*}
		\ket{\psi(t)} = \alpha\ket{+} \exp\left(-\frac{iE_+ t}{\hbar}\right) + \beta\ket{-} \exp\left(-\frac{iE_- t}{\hbar}\right)
	\end{equation*}
	
	We have initial condition $\ket{\psi(0)} = \ket{\textnormal{e, 0}}$ so:
	\begin{align*}
		\frac{\alpha}{\sqrt{2}} + \frac{\beta}{\sqrt{2}} = 1
		& &
		\frac{\alpha}{\sqrt{2}} - \frac{\beta}{\sqrt{2}} = 0
	\end{align*}
	\begin{equation*}
		\Rightarrow \alpha = \beta = \frac{1}{\sqrt{2}}
	\end{equation*}
	
	Thus
	\begin{align*}
		\ket{\psi(t)} &= \frac{1}{2} \left\{\exp\left[-i\left(\frac{\omega+\Omega}{2}\right)t\right]\ket{+} \;+\; \exp\left[-i\left(\frac{\omega-\Omega}{2}\right)t\right]\ket{-}\right\} \\
		&= \exp\left(-\frac{i\omega}{2}t\right) \left[\cos\left(\frac{\Omega}{2}t\right)\ket{\textnormal{e, 0}} + i\sin\left(\frac{\Omega}{2}t\right)\ket{\textnormal{g, 1}}\right]
	\end{align*}
	
	\part From above, we have probability of excited state:
	\begin{equation*}
		P_{\ket{\textnormal{e, 0}}}(t) = \cos^2 \left(\frac{\Omega}{2}t\right)
	\end{equation*}
	
	For $P > 0.9$, we have:
	\begin{align*}
		\cos^2 \left(\frac{\Omega}{2}t\right) &> 0.9 \\
		\frac{\Omega}{2}t &< \arccos\sqrt{0.9} = 0.322 \\
		t &< \frac{0.644}{\Omega}
	\end{align*}
	
	\part To measure 1 photon, we need state $\ket{\textnormal{g, 1}}$:
	\begin{equation*}
		P_{\ket{\textnormal{g, 1}}}(t) = \sin^2\left(\frac{\Omega}{2}t\right)
	\end{equation*}
	
	After such measurement, the dressed states shall collapse into $\ket{\textnormal{g, 1}}$ immediately.
	
	\part Without the measurement result, we have a probabilistic description given by the density matrix below:
	\begin{align*}
		\rho &= P_{\ket{\textnormal{e, 0}}}(t) \ket{\textnormal{e, 0}}\bra{\textnormal{e, 0}} + P_{\ket{\textnormal{g, 1}}}(t) \ket{\textnormal{g, 1}}\bra{\textnormal{g, 1}} \\
		&= \bordermatrix{
			& \bra{\textnormal{e}} & \bra{\textnormal{g}} \cr
			& \cos^2 \dfrac{\Omega t}{2} & 0 \cr
			& 0 & \sin^2 \dfrac{\Omega t}{2}
		}
	\end{align*}
	
	From Boltzmann distribution we have:
	\begin{align*}
		\frac{N_\textnormal{e}}{N_\textnormal{g}} &= \exp\left(-\frac{\hbar\omega}{k_\textnormal{B} T}\right) \\
		\Rightarrow p_\textnormal{e} &= \frac{N_\textnormal{e}}{N_\textnormal{e} + N_\textnormal{g}} \\
		&= \frac{N_{e}}{N_{e}\left[1+\exp\left(\dfrac{\hbar\omega}{k_\textnormal{B} T}\right)\right]} \\
		&= \frac{1}{1+\exp\left(\dfrac{\hbar\omega}{k_\textnormal{B} T}\right)} \\
		\Rightarrow p_\textnormal{g} &= 1-p_\textnormal{e} \\
		&= \frac{1}{1+\exp\left(-\dfrac{\hbar\omega}{k_\textnormal{B} T}\right)}
	\end{align*}
	
	Equating $p_\textnormal{g}$ with $P_{\ket{\textnormal{g, 1}}}$ gives:
	\begin{gather*}
		\sin^2 \frac{\omega t}{2} = \frac{1}{1+\exp\left(-\dfrac{\hbar\omega}{k_\textnormal{B} T}\right)} \\
		\Rightarrow 1+\exp\left(-\frac{\hbar\omega}{k_\textnormal{B} T}\right) = \frac{1}{\sin^2 \frac{\Omega t}{2}} \\
		\frac{\hbar\omega}{k_\textnormal{B} T} = -\ln\left( \frac{1}{\sin^2 \frac{\Omega t}{2}} - 1 \right) \\
		= \ln\left(\frac{\sin^2 \frac{\Omega t}{2}}{1 - \sin^2 \frac{\Omega t}{2}}\right) \\
		\Rightarrow T = \frac{\hbar\omega}{k_\textnormal{B} \ln\left(\tan^2 \dfrac{\Omega t}{2}\right)}
	\end{gather*}
	where $T$ is the critical temperature at which both probabilities match.
	
	However note that with Boltzmann distribution, this equality only holds for $t$ where $0 \le P_{\ket{\textnormal{g, 1}}} \le \diagfrac{1}{2}$ for all $T$.
\end{parts}