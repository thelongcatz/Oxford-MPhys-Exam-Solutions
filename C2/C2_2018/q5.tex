One of the rare cases where the question on NMR quantum computing is relatively easy to answer.

\begin{parts}
	\part Zeeman energy for spin:
	\begin{equation*}
		E = \hbar\gamma B = \hbar\omega
	\end{equation*}
	
	NMR Hamiltonian:
	\begin{equation*}
		\mathcal{H} = \hbar\omega_1\frac{\sigma_{1z}}{2} + \hbar\omega_2\frac{\sigma_{2z}}{2} + \hbar\omega_{12}\frac{\sigma_{1z}\sigma_{2z}}{4}
	\end{equation*}
	
	At thermal equilibrium, the kinetic population shall be described by the Boltzmann distribution:
	\begin{gather*}
		\frac{n_\uparrow}{n_\downarrow} = \mathrm{e}^{-\hbar\omega / \boltzmann T} \\
		\Rightarrow
		\begin{cases}
			p_{\uparrow\uparrow} = \frac{1}{4} + \mathrm{e}^{-2\hbar\omega / \boltzmann T} \approx \frac{1}{4} - \frac{2\hbar\omega}{\boltzmann T} \\
			p_{\downarrow\downarrow} = \frac{1}{4} + \frac{2\hbar\omega}{\boltzmann T}
		\end{cases} \mtext{to 1st order}
	\end{gather*}
	
	Thus we have density matrix:
	\begin{align*}
		\rho_\textnormal{th} &= \frac{1}{4} \ket{\uparrow\downarrow}\bra{\uparrow\downarrow} + \frac{1}{4} \ket{\downarrow\uparrow}\bra{\downarrow\uparrow} + p_{\uparrow\uparrow} \ket{\uparrow\uparrow}\bra{\uparrow\uparrow} + p_{\downarrow\downarrow} \ket{\downarrow\downarrow}\bra{\downarrow\downarrow} \\
		&= \frac{1}{4} \begin{pmatrix}
			1+\frac{\hbar\omega}{\boltzmann T} & 0 & 0 & 0 \\
			0 & 1 & 0 & 0 \\
			0 & 0 & 1 & 0 \\
			0 & 0 & 0 & 1-\frac{\hbar\omega}{\boltzmann T}
		\end{pmatrix}
	\end{align*}
	
	\part From the question, we have $\mathtt{CTG} \equiv \mathtt{crush} \cdot \ket{1}\bra{1} \otimes \mathrm{X}_\theta$.
	
	We then have:
	\begin{align*}
		\mathrm{X}_\theta &= \mathrm{e}^{-i\theta\sigma_x / 2} \\
		&= \mathds{1} - i\theta \cdot \frac{1}{2}
		\begin{pmatrix}
			0 & 1 \\
			1 & 0
		\end{pmatrix}
		+ \frac{1}{2} \cdot \rbracket{\frac{i\theta}{2}}^2
		\begin{pmatrix}
			1 & 0 \\
			0 & 1
		\end{pmatrix}
		+ \ldots \\
		&= \mathds{1} \cos\frac{\theta}{2} - i\sigma_x \sin\frac{\theta}{2} \\
		&= \begin{pmatrix}
			\cos\frac{\theta}{2} & -i\sin\frac{\theta}{2} \\
			-i\sin\frac{\theta}{2} & \cos\frac{\theta}{2}
		\end{pmatrix}
	\end{align*}
	
	Hence controlled-$\mathrm{X}_\theta$ may be written as:
	\begin{equation*}
		\mathrm{CX}_\theta = \begin{pmatrix}
			1 & 0 & 0 & 0 \\
			0 & 1 & 0 & 0 \\
			0 & 0 & \multicolumn{2}{c}{\multirow{2}{*}{$\mathrm{X}_\theta$}} \\
			0 & 0 & \multicolumn{2}{c}{}
		\end{pmatrix}
	\end{equation*}
	
	We thus have the following CP map:
	\begin{align*}
		\rho &\rightarrow (\mathrm{CX}_\theta) \rho (\mathrm{CX}_\theta)^\dag \\
		&= \begin{pmatrix}
			1 & & & \\
			& 1 & & \\
			& & \multicolumn{2}{c}{\multirow{2}{*}{$\mathrm{X}_\theta$}} \\
			& & \multicolumn{2}{c}{}
		\end{pmatrix}
		\begin{pmatrix}
			a & & & \\
			& b & & \\
			& & c & \\
			& & & d
		\end{pmatrix}
		\begin{pmatrix}
			1 & & & \\
			& 1 & & \\
			& & \multicolumn{2}{c}{\multirow{2}{*}{$\mathrm{X}_\theta^\dag$}} \\
			& & \multicolumn{2}{c}{}
		\end{pmatrix} \\
		&= \begin{pmatrix}
			1 & & & \\
			& 1 & & \\
			& & \cos\frac{\theta}{2} & -i\sin\frac{\theta}{2} \\
			& & -i\sin\frac{\theta}{2} & \cos\frac{\theta}{2}
		\end{pmatrix}
		\begin{pmatrix}
			a & & & \\
			& b & & \\
			& & c & \\
			& & & d
		\end{pmatrix}
		\begin{pmatrix}
			1 & & & \\
			& 1 & & \\
			& & \cos\frac{\theta}{2} & i\sin\frac{\theta}{2} \\
			& & i\sin\frac{\theta}{2} & \cos\frac{\theta}{2}
		\end{pmatrix} \\
		&= \ldots \begin{pmatrix}
			a & & & \\
			& b & & \\
			& & c \cos\frac{\theta}{2} & ic \sin\frac{\theta}{2} \\
			& & id \sin\frac{\theta}{2} & d \cos\frac{\theta}{2}
		\end{pmatrix} \\
		&= \begin{pmatrix}
			a & & & \\
			& b & & \\
			& & c \cos^2\frac{\theta}{2} + d \sin^2\frac{\theta}{2} & \ldots \\
			& & \ldots & c \sin^2\frac{\theta}{2} + d \cos^2\frac{\theta}{2}
		\end{pmatrix}
		\xRightarrow{\mathtt{crush}} \begin{pmatrix}
			a & & & \\
			& b & & \\
			& & c^\prime & \\
			& & & d^\prime
		\end{pmatrix}
	\end{align*}
	
	Thus we have:
	\begin{align*}
		c^\prime &= c \cos^2\frac{\theta}{2} + d \sin^2\frac{\theta}{2} \\
		&= c \rbracket{\frac{1 + \cos\theta}{2}} + d \rbracket{\frac{1 - \cos\theta}{2}} \\
		&= \frac{(c+d) + \cos\theta (c-d)}{2}
	\end{align*}
	
	Similarly we have $d^\prime = \dfrac{(c+d) - \cos\theta (c-d)}{2}$.
	
	If the upper qubit is acted upon instead, we simply swap $b$ and $c^\prime$.
	
	$a$ is unaffected since $\mathtt{CTG}$ is a conditional gate and $a$ corresponds to the case where both qubits are $0$.
	
	\part For $\rho_\textnormal{th}$, we have $a \equiv \diagfrac{1}{4} (1+\epsilon)$, $b \equiv \diagfrac{1}{4} \equiv c$, $d \equiv \diagfrac{1}{4} (1-\epsilon)$.
	
	After $\mathtt{CTG}(\theta_1 = \arccos(\diagfrac{1}{3}), \textnormal{lower})$, we have density matrix:
	\begin{equation*}
		\begin{pmatrix}
			a & & & \\
			& b & & \\
			& & c^\prime & \\
			& & & d^\prime
		\end{pmatrix}
	\end{equation*}
	where $c^\prime = \dfrac{(c+d) + \diagfrac{1}{3} (c-d)}{2}$, $d^\prime = \dfrac{(c+d) - \diagfrac{1}{3} (c-d)}{2}$.
	
	After $\mathtt{CTG}(\theta_2 = \pi/2, \textnormal{upper})$, we have:
	\begin{equation*}
		\begin{pmatrix}
			a & & & \\
			& b^\prime & & \\
			& & c^\prime & \\
			& & & d^{\prime\prime}
		\end{pmatrix}
	\end{equation*}
	where $b^\prime = \dfrac{(b+d^\prime) + 0 (b-d^\prime)}{2}$, $d^{\prime\prime} = \dfrac{(b+d^\prime) - 0 (b-d^\prime)}{2}$.
	
	Simplifying the expressions then gives:
	\begin{align*}
		c^\prime &= \frac{2}{3}c + \frac{1}{3}d = \frac{1}{6} + \frac{1}{12}(1-\epsilon) \\
		d^\prime &= \frac{1}{3}c + \frac{2}{3}d = \frac{1}{12} + \frac{1}{6}(1-\epsilon) \\
		b^\prime &= d^{\prime\prime} = \frac{b+d^\prime}{2} = \frac{1}{6} + \frac{1}{12}(1-\epsilon)
	\end{align*}
	
	So we have the resulting density matrix:
	\begin{align*}
		\rho_\textnormal{th} &= \begin{pmatrix}
			\dfrac{1}{4}(1+\epsilon) & & & \\
			& \multicolumn{3}{c}{\multirow{3}{*}{diag$\rbracket{\dfrac{1}{6} + \dfrac{1}{12}(1-\epsilon)}$}} \\
			& \multicolumn{3}{c}{} \\
			& \multicolumn{3}{c}{}
		\end{pmatrix} \\
		&= \textnormal{diag}\rbracket{\frac{1}{6} + \frac{1}{12}(1-\epsilon)} + \begin{pmatrix}
			\dfrac{1}{12} - \dfrac{1}{12} + \dfrac{\epsilon}{4} + \dfrac{\epsilon}{12} & & & \\
			& 0 & & \\
			& & 0 & \\
			& & & 0
		\end{pmatrix} \\
		&= \frac{1}{4} \textnormal{diag}\rbracket{1-\frac{\epsilon}{3}} + \begin{pmatrix}
			\dfrac{\epsilon}{3} & & & \\
			& 0 & & \\
			& & 0 & \\
			& & & 0
		\end{pmatrix}
	\end{align*}
	
	Thus $p = \epsilon/3 = \hbar\omega/3\boltzmann T$.
	
	Now we note that $p$ scales as the inverse of the system size ($\rho$ has dimension $2^n$): $p\propto\frac{1}{2^n}$.
	Therefore scaling up the system decreases $p$ quickly, rendering large scale NMR computing via pseudo-pure state unfeasible.
	
	\part To interact with the NMR system, one may employ an RF pulse at the appropriate $\omega$ and applying it for some time $\tau$ that corresponds to the desired phase.
	In addition, by rotating the polarisation of the RF pulse, one may construct a general gate.
	
	The catch with homonuclear system is that the transition energies are exactly the same, so spatial separation is required for qubit addressing.
	Whereas for heteronuclear system, there are only a finite number of elements, which then imposes an upper limit to the number of qubits available.
	
	NMR computing is not scalable due to the ever shrinking effective purity, as explored in the previous part.
\end{parts}