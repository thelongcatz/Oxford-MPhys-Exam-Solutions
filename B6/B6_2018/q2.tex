\draft
\begin{parts}
	\part Chemical potential: energy required to add a particle intro a system under thermal equilibrium at constant entropy and volume.
	
	Fermi energy: chemical potential at temperature $T=\SI{0}{\kelvin}$.
	Also in the context of CMP, halfway point before the highest filled state and lowest unfilled state if state distribution discontinuous.
	
	\part TISE gives $\hat{H}\ket{\psi} = E\ket{\psi}$
	
	System wavefunction $\ket{\psi} = \ldots + \ket{n} + \ket{n+1} + \ldots$
	
	Try $\hat{H}(\ket{n}) = \varepsilon_0 \ket{n} + (-t) \ket{n-1} + (-t) \ket{n+1}$
	
	For plane wave,
	\begin{align*}
		\bra{\mathbf{r}}\ket{n} &= A e^{ikna} \\
		\Rightarrow \hat{H}\ket{n} &= E_n \ket{n} \\
		\Rightarrow E_n A e^{ikna} &= \varepsilon_0 A e^{ikna} - t A e^{ik(n-1)a} - t A e^{ik(n+1)a} \\
		\Rightarrow E_n &= \varepsilon_0 - t \rbracket{e^{ika} + e^{-ika}} \\
		&= \varepsilon_0 - 2t\cos\rbracket{ka}
	\end{align*}
	
	Born-von Karman boundary condition:
	\begin{align*}
		e^{ikna} &= e^{ik(n+1)a} \\
		\Rightarrow e^{ika} &= 1 \\
		\Rightarrow k &= \frac{2\pi}{a}
	\end{align*}
	
	Dispersion relation in the 1st Brillouin Zone\footnote{Tip: trace the curve and check you have repeated within a zone}:
	\image{.8\linewidth}{q2-dispersion}
	
	Also we have:
	\begin{align*}
		\deri{E}{k} &= 2ta \sin\rbracket{ka} \\
		&= a\sqrt{4t^2 - (\varepsilon_0 - E)^2}
	\end{align*}
	Density of states:
	\begin{align*}
		\deri{N}{E} &= \deri{N}{k} \cdot \deri{k}{E} \\
		&= \frac{Na}{2\pi} \cdot \underbracket{2}_{\mathclap{\textnormal{spin}}} \cdot \overbracket{2}^{\mathclap{\pm\inftsml{k}\textnormal{degeneracy}}} \cdot \deri{k}{E} \mtext{assuming ``isotropy'' in 1D} \\
		&= \frac{2Na}{\pi} \cdot \frac{1}{a\sqrt{4t^2 - (\varepsilon_0 - E)^2}} \\
		\Rightarrow g(E) \inftsml{E} &= \frac{2N}{\pi} \frac{1}{\sqrt{4t^2 - (\varepsilon_0 - E)^2}} \inftsml{E}
	\end{align*}
	\image{.7\linewidth}{q2-dos}
	
	\part Assuming monovalent atoms, the Fermi energy is simply:
	\begin{equation*}
		E(ka=\frac{\pi}{2}) = \varepsilon_0
	\end{equation*}
	
	Chemical potential:
	\begin{align*}
		\mu &= \deri{E}{N} = \frac{1}{g(E)} \\
		&= \frac{\pi}{2N} \sqrt{4t^2 - (\varepsilon_0 - E)^2} \not\propto T
	\end{align*}
	\image{.4\linewidth}{q2-symmetric-e}
	Since $E$ is \underline{symmetric} about $E_0$:
	\begin{equation*}
		\abs{\avg{E_\mathrm{e} - \varepsilon_0}} = \abs{\avg{E_\mathrm{h} - \varepsilon_0}}
	\end{equation*}
	So $\mu$ remains constant ($\varepsilon_0$).
	
	Or:
	\begin{align}
		N &= \defint{\varepsilon_0 - 2t}{\varepsilon_0 + 2t}{g(E) \frac{1}{e^{\beta(E-\mu)} + 1}}{E} \label{eqn:q2-N-1} \\
		&= \defint{\varepsilon_0 - 2t}{\varepsilon_0 + 2t}{g(E) \sbracket{1 - \frac{1}{e^{\beta(\mu - E)} + 1}}}{E} \notag \\
		&= \underbracket{\indefint{g(E)}{E}}_{\mathclap{\textnormal{total \# of states}=2N}} - \indefint{\frac{g(E)}{e^{\beta(\mu - E)} + 1}}{E} \notag \\
		\Rightarrow N &= \indefint{\frac{g(E)}{e^{\beta(\mu - E)} + 1}}{E} \label{eqn:q2-N-2}
	\end{align}
	
	Comparing \eqref{eqn:q2-N-1} and \eqref{eqn:q2-N-2}:
	\begin{equation*}
		\indefint{g(E) \frac{1}{e^{\beta(E-\mu)} + 1}}{E} = \indefint{\frac{g(E)}{e^{\beta(\mu - E)} + 1}}{E}
	\end{equation*}
	Nothing that this must be true for all $T$, this implies that $\mu$ must be independent of $T$.
	
	\part \image{.7\linewidth}{q2-energy-surface-sq}
	Monovalent $\Rightarrow$ $\frac{1}{2}$ of BZ area occupied by spin degeneracy.
	
	By symmetry, (noting that the previous part has no dependence on dimensions)
	\begin{equation*}
		\mu = E_F = \varepsilon_0 \mtext{independent of $T$}
	\end{equation*}
	
	\part Now:
	\begin{equation*}
		E(k_x, k_y) = \varepsilon_0 - 2t_x \cos (k_x a_x) - 2t_y \cos (k_y a_y)
	\end{equation*}
	
	Also:
	\begin{align*}
		k_x^\textnormal{max} &= \pm\frac{\pi}{a_x} \\
		k_y^\textnormal{max} &= \pm\frac{\pi}{a_y} \\
		a_x &< a_y \Rightarrow k_x^\textnormal{max} > k_y^\textnormal{max}
	\end{align*}
	We expect BZ and Fermi surface to be squashed towards $x$ axis as $t_x > t_y$.
	\image{.8\linewidth}{q2-energy-surface-rect}
\end{parts}